\documentclass[a4paper,oneside]{article}
\usepackage[utf8]{inputenc}
\usepackage{type1cm}
\usepackage[italian]{babel}
\usepackage[%hypertex,
                 unicode=true,
                 plainpages = false, 
                 pdfpagelabels, 
                 bookmarks=true,
                 bookmarksnumbered=true,
                 bookmarksopen=true,
                 breaklinks=true,
                 backref=false,
                 colorlinks=true,
                 linkcolor = blue,		% Use "blue" if you want to highlight them
                 urlcolor  = blue,
                 citecolor = red,
                 anchorcolor = green,
                 hyperindex = true,
                 linktocpage = true,
                 hyperfigures
]{hyperref}
\usepackage{amsthm}
\renewenvironment{proof}
	{\noindent\textsc{Soluzione:}}
	{\begin{flushright}$\blacksquare$\end{flushright}\vskip 1em}
\usepackage{amsmath}
\usepackage{amsfonts}
\usepackage{amssymb}
\usepackage{cancel}
\usepackage{graphicx}
\usepackage{asymptote}
\usepackage{geometry}
\author{A.V.}
\title{Esercizi svolti di Fisica Generale II\\
	\large{\textit{Dalle lezioni di Giuseppe Dalba}}}
\newtheorem{problema}{Problema}
\let\oldhat\hat
\renewcommand{\vec}[1]{\mathbf{#1}}
\renewcommand{\hat}[1]{\oldhat{\mathbf{#1}}}

\begin{document}

\maketitle

\begin{problema}
	Preso un filo sottile carico, di lunghezza $2a$ e 
	distribuzione lineare di carica $\lambda = \mathrm{costante}$, 
	determinare:
	\begin{enumerate}
		\item $\vec{E} = \vec{E}(x,\,y,\,z)$ in un qualsiasi punto 
		che si trovi sull'asse del filo.
		\item Sempre in un generico punto sull'asse, trovare $\vec{E}$ 
		nel limite in cui $a \longrightarrow \infty$.
	\end{enumerate}
\end{problema}
\begin{proof}
	\begin{enumerate}
		\item Da considerazioni di simmetria (vedi Figura \ref{fig:filo_sottile_carico}),
		vale che $\vec{E} \equiv \vec{E}_x$,
		dove $\vec{E}_x$ indica la componente del campo $\vec{E}$ nella direzione $x$.
		\begin{figure}
			\centering
			\begin{asy}
				import graph;
				import cse5;
				texpreamble("\let\oldhat\hat
				\renewcommand{\vec}[1]{\mathbf{#1}}
				\renewcommand{\hat}[1]{\oldhat{\mathbf{#1}}}");
				size(8cm);
				real xmin=-2,xmax=12;
				real ymin=-8,ymax=8;
				xaxis(Label("\small $x$",position=EndPoint,align=SE),
				xmin, xmax, arrow=EndArrow, NoTicks);
				yaxis(Label("\small $y$",position=EndPoint,align=NW),
				ymin, ymax, arrow=EndArrow, NoTicks);
				dot(Label("$P(x,\,y,\,z)$",align=NNE),(10,0));
				draw("$\vec{r}$",(0,0)--(10,0),1bp+black,Arrow(10bp,8));
				draw(Label("$\vec{r}-\vec{r'}$",position=MidPoint,NE),
				(0,4)--(10,0),1bp+black,Arrow(8bp,8));
				draw(Label("$\vec{r'}$",position=MidPoint,W),
				(-0.2,0)--(-0.2,4),1bp+black,Arrow(8bp,8));
				draw((0,-6)--(0,6),solid+1.5bp);
				MarkAngle("\vartheta",black,(0,4),(10,0),(0,0),2,black);
				labely("$a$",6,W);
				labely("$-a$",-6,W);
			\end{asy}
			\caption{Filo sottile carico}
			\label{fig:filo_sottile_carico}
		\end{figure}
		Ora,
		$$
		E_x = k_e \int_{l} \frac{\lambda dl (x-x')}{\left[ (x-x')^2 + (y-y')^2 + 
		(z-z')^2 \right]^{3/2}}
		$$
		Poiché $\vec{r} = (x,\,0,\,0)$, $\vec{r'} = (0,\,y',\,0)$ e $dl \equiv dy'$,
		si ottiene che
		$$
		E_x = k_e \int_{-a}^{a} \frac{\lambda dy' x}{\left[ x^2 + y'^2 \right]^{3/2}}
		= k_e \lambda x \int_{-a}^{a} \frac{dy'}{\left[ x^2 + y'^2 \right]^{3/2}}
		$$
		Usando la semplificazione $y' = x\tan\vartheta, \; dy' = 
		\frac{x}{\cos^2\vartheta}d\vartheta$, abbiamo:
		\begin{align*}
		E_x &= k_e \lambda \cancel{x} \int_{\arctan(-a/x)}^{\arctan(a/x)}
		\frac{\cancel{x} d\vartheta}{\cancel{cos^2\vartheta}} 
		\frac{cos^{\cancel{3}^1}\vartheta}{x^{\cancel{3}^1}}\\
		&= \frac{k_e \lambda}{x} \int_{\arctan(-a/x)}^{\arctan(a/x)}
		\cos\vartheta d\vartheta\\ &= 
		\boxed{\frac{2 k_e \lambda}{x} \sin(\arctan(a/x))}
		\end{align*}
		\item Per $a \longrightarrow \infty$, $\arctan(a/x) \longrightarrow \pi/2$.
		Quindi,
		$$
		E \longrightarrow \frac{2 k_e \lambda}{x} \sin\left(\frac{\pi}{2}\right)
		= \frac{2 k_e \lambda}{x} = \boxed{\frac{\lambda}{2 \pi \varepsilon_0 x}}
		$$
		Notiamo che il campo ha lo stesso andamento riscontrato nel caso di 
		cariche puntiformi! Inoltre, per ragioni di natura pratica, conviene 
		equivalentemente studiare $x \ll a$ invece che $a \longrightarrow \infty$.
 	\end{enumerate}
\end{proof}

\begin{problema}
	Presa una spira sottile carica, di raggio $R$ e 
	distribuzione lineare di carica $\lambda = \mathrm{costante}$, 
	determinare:
	\begin{enumerate}
		\item $\vec{E} = \vec{E}(x,\,y,\,z)$ in un qualsiasi punto 
		che si trovi sull'asse della spira.
		\item $\vec{E} = \vec{E}(x,\,y,\,z)$ come nel caso precedente,
		ma considerando i limiti $x \ll 1$ e $x \gg R$.
		\item $\vec{E} = \vec{E}(x,\,y,\,z)$ in un qualsiasi punto 
		che si trovi sull'asse di un disco avente lo stesso raggio
		e densità di carica superficiale $\sigma = \mathrm{costante}$.
		\item $\vec{E} = \vec{E}(x,\,y,\,z)$ come nel caso precedente,
		ma considerando il limite $R \longrightarrow \infty$ (piano infinito).
	\end{enumerate}
\end{problema}
\begin{proof}
	\begin{enumerate}
		\item La soluzione è identica al problema del filo sottile carico: 
		chiamando $x$ l'asse della spira, $\vec{E} \equiv \vec{E}_x$. Considerando
		che $\vec{r} = (x,\,0,\,0)$ e $\vec{r'} = (0,\,y',\,z')$, il calcolo diventa:
		\begin{align*}
		E_x &= k_e \lambda \int_l \frac{dl(x-x')}{|\vec{r} - \vec{r'}|^3}\\
		&= \frac{k_e \lambda x}{|\vec{r} - \vec{r'}|^3} \int_l dl\\
		&= \frac{k_e \lambda x l}{|\vec{r} - \vec{r'}|^3}\\
		&= \boxed{\frac{k_e \lambda x 2\pi R}{(x^2 + R^2)^{3/2}}}
		\end{align*}
		\item Se $x \ll 1$, $E \longrightarrow 0$. Se $x \gg R$, abbiamo invece
		$$
		E \sim \frac{k_e\cancel{x}\lambda 2 \pi R}{x^{\cancel{3}^2}} =
		\frac{k_e\lambda 2 \pi R}{x^2} =
		\frac{1}{\null_2\cancel{4} \cancel{\pi} \varepsilon_0} 
		\frac{k_e\lambda \cancel{2} \cancel{\pi} R}{x^2} =
		\frac{\lambda R}{2 \varepsilon_0 x^2}
		$$
		Se $Q$ è la carica del filo, $Q = \lambda 2 \pi R$, quindi
		$$
		E \sim \boxed{\frac{k_e Q}{x^2}}
		$$
		Il risultato ci dice che, da lontano, la spira è assimilabile 
		ad una carica puntiforme!
		\item Per calcolare il campo generato da un disco, conviene prima
		calcolare il campo $dE_x$ generato da un anello sottile, di spessore infinitesimo
		$dr'$. La superficie infinitesima dell'anello sarà dunque $dS = dr'dl$
		\footnote{
		Equivalentemente, è possibile definire $dS = 2 \pi r' dr'$, cioè prendere
		una corona circolare infinitesima, e integrare direttamente.		
		}.
		Abbiamo:
		\begin{align*}
		dE_x &= k_e \int_S \frac{\sigma dS (x-x')}{|\vec{r} - \vec{r'}|^3}
		= k_e \int_l \frac{\sigma dr'dl x}{(x^2+r'^2)^{3/2}}\\
		&= \frac{k_e \sigma dr' x}{(x^2+r'^2)^{3/2}} \int_l dl
		= \frac{k_e \sigma dr' x l}{(x^2+r'^2)^{3/2}}\\
		&= \frac{k_e \sigma dr' x 2 \pi r'}{(x^2+r'^2)^{3/2}}
		= \frac{\sigma x r' dr'}{2 \varepsilon_0 (x^2+r'^2)^{3/2}}
		\end{align*}
		Ora, basta integrare su tutta la lunghezza del raggio:
		\begin{align*}
		E_x &= \int_{0}^{R} dE_x = \int_{0}^{R} 
		\frac{\sigma x r' dr'}{2 \varepsilon_0 (x^2+r'^2)^{3/2}}\\
		&= \frac{\sigma x}{2 \varepsilon_0} \frac{1}{2} \int_{0}^{R} 
		\frac{2 r' dr'}{(x^2+r'^2)^{3/2}}\\
		&= \frac{1}{\cancel{2}} \frac{\sigma x}{2 \varepsilon_0}
		\left. \frac{(-\cancel{2})}{\sqrt{x^2+r'^2}} \right\lvert_O^R\\
		&= \boxed{\frac{\sigma}{2 \varepsilon_0} \left( 1 - 
		\frac{x}{\sqrt{x^2+R^2}}\right)}
		\end{align*}
		\item Per $x \ll R$, dal caso precedente segue immediatamente che:
		$$
		\boxed{E \sim \frac{\sigma}{2 \varepsilon_0}}
		$$
		A distanza ravvicinata dal disco, $\vec{E}$ si comporta come un campo costante!
	\end{enumerate}
\end{proof}

\begin{problema}
	Calcolare $\vec{E}$ in un generico punto del piano in Figura
	\ref{fig:distribuzioni_parallele}, dove le distribuzioni superficiali 
	di carica schematizzate sono da considerare di lunghezza e larghezza
	infinite.
	\begin{figure}
			\centering
			\begin{asy}
				import graph;
				import markers;
				texpreamble("\let\oldhat\hat
				\renewcommand{\vec}[1]{\mathbf{#1}}
				\renewcommand{\hat}[1]{\oldhat{\mathbf{#1}}}");
				size(8cm);
				real xmin=-8,xmax=8;
				real ymin=-6,ymax=6;
				xaxis(Label("\small $x$",position=EndPoint,align=SE),
				xmin, xmax, arrow=EndArrow, NoTicks);
				yaxis(Label("\small $y$",position=EndPoint,align=NW),
				ymin, ymax, arrow=EndArrow, NoTicks);
				draw(Label("\small $+\sigma$",position=EndPoint,N,p=black),
				(-3,-6)--(-3,6),invisible,
				CrossIntervalMarker(12,4,size=2,angle=45,bp+black,
				dotframe(invisible)));
				draw(Label("\small $+\sigma$",position=EndPoint,N,p=black),
				(-1,-6)--(-1,6),invisible,
				CrossIntervalMarker(12,4,size=2,angle=45,bp+black,
				dotframe(invisible)));
				draw(Label("\small $+\sigma$",position=EndPoint,N,p=black),
				(1,-6)--(1,6),invisible,
				CrossIntervalMarker(12,4,size=2,angle=45,bp+black,
				dotframe(invisible)));
				draw(Label("\small $+\sigma$",position=EndPoint,N,p=black),
				(3,-6)--(3,6),invisible,
				CrossIntervalMarker(12,4,size=2,angle=45,bp+black,
				dotframe(invisible)));
				draw(Label("\small $a$",align=Center,filltype=UnFill),
				(-2.9,-4)--(-1.1,-4),arrow=Arrows(4bp,8));
				draw(Label("\small $a$",align=Center,filltype=UnFill),
				(-0.9,-4)--(0.9,-4),arrow=Arrows(4bp,8));
				draw(Label("\small $a$",align=Center,filltype=UnFill),
				(1.1,-4)--(2.9,-4),arrow=Arrows(4bp,8));
			\end{asy}
			\caption{Distribuzioni di carica parallele}
			\label{fig:distribuzioni_parallele}
		\end{figure}
\end{problema}
\begin{proof}
	Da considerazioni di natura geometrica, si vede chiaramente che il campo
	assume i valori riportati in Figura \ref{fig:distribuzioni_parallele_soluzione}.
	\begin{figure}
			\centering
			\begin{asy}
				import graph;
				import markers;
				texpreamble("\let\oldhat\hat
				\renewcommand{\vec}[1]{\mathbf{#1}}
				\renewcommand{\hat}[1]{\oldhat{\mathbf{#1}}}");
				size(8cm);
				real xmin=-8,xmax=8;
				real ymin=-6,ymax=6;
				xaxis(Label("\small $x$",position=EndPoint,align=SE),
				xmin, xmax, arrow=EndArrow, NoTicks);
				yaxis(Label("\small $y$",position=EndPoint,align=NW),
				ymin, ymax, arrow=EndArrow, NoTicks);
				draw(Label("\small $+\sigma$",position=EndPoint,N,p=black),
				(-3,-6)--(-3,6),invisible,
				CrossIntervalMarker(12,4,size=2,angle=45,bp+black,
				dotframe(invisible)));
				draw(Label("\small $+\sigma$",position=EndPoint,N,p=black),
				(-1,-6)--(-1,6),invisible,
				CrossIntervalMarker(12,4,size=2,angle=45,bp+black,
				dotframe(invisible)));
				draw(Label("\small $+\sigma$",position=EndPoint,N,p=black),
				(1,-6)--(1,6),invisible,
				CrossIntervalMarker(12,4,size=2,angle=45,bp+black,
				dotframe(invisible)));
				draw(Label("\small $+\sigma$",position=EndPoint,N,p=black),
				(3,-6)--(3,6),invisible,
				CrossIntervalMarker(12,4,size=2,angle=45,bp+black,
				dotframe(invisible)));
				label(rotate(90)*"\small $E = -\frac{2\sigma}{\varepsilon_0}$",
				(-4,0),NoAlign,p=solid+black,filltype=UnFill);
				label(rotate(90)*"\small $E = -\frac{\sigma}{\varepsilon_0}$",
				(-2,0),NoAlign,p=solid+black,filltype=UnFill);
				label(rotate(90)*"\small $E = 0$",
				(0,0),NoAlign,p=solid+black,filltype=UnFill);
				label(rotate(90)*"\small $E = \frac{\sigma}{\varepsilon_0}$",
				(2,0),NoAlign,p=solid+black,filltype=UnFill);
				label(rotate(90)*"\small $E = \frac{2\sigma}{\varepsilon_0}$",
				(4,0),NoAlign,p=solid+black,filltype=UnFill);
			\end{asy}
			\caption{Distribuzioni di carica parallele}
			\label{fig:distribuzioni_parallele_soluzione}
		\end{figure}
\end{proof}

\begin{problema}
	Calcolare il campo $\vec{E}$ generato da un piano infinito carico 
	elettricamente, considerandolo come successione infinita di fili di 
	lunghezza a loro volta infinita. Il piano ha una densità
	superficiale di carica $\sigma = \mathrm{costante}$.
\end{problema}
\begin{proof}
	Iniziamo con delle considerazioni di natura geometrica. Prendiamo il piano
	di cariche in modo tale che sia coincidente col piano $xz$ e dividiamolo
	in tanti fili paralleli all'asse $z$, ognuno di spessore infinitesimo $dx$. 
	Dalla condizione $\sigma = \mathrm{costante}$, otteniamo subito che la 
	densità lineare di carica dei fili $\lambda = \sigma dx = \mathrm{costante}$.
	
	Siano ora $x'$ la distanza di un filo generico dall'asse $z$ e $y$ la distanza 
	di un punto generico $P$ dal piano di cariche. Per semplicità, possiamo prendere 
	$\vec{r'} = (x',\,0,\,0)$ e $\vec{r} = (0,\,y,\,0)$, in modo tale che
	$P$ abbia coordinate $(0,\,y,\,0)$ e $dx = dx'$. Il campo nel punto $P$ generato 
	dal filo in posizione $\vec{r'}$ non sarà, in generale, parallelo all'asse $y$. 
	Avrà una componente $E_x$ diretta lungo l'asse $x$ e una componente
	$E_y$ diretta lungo l'asse $y$. 
	Ma se prendiamo il filo in posizione $-\vec{r'}$, quest'ultimo indurrà in $P$
	un campo che avrà componenti $-E_x$ e $E_y$! Quindi il campo totale generato
	dai due fili sarà diretto lungo l'asse $y$ e avrà modulo\footnote{
	Per il campo generato da un filo infinitamente lungo, 
	riguardare il relativo esercizio.	
	}
	$$
	E_{\text{2 fili}} = 2E_y = \frac{\lambda y}{\pi \varepsilon_0 (x'^2 + y^2)}
	$$
	Ora, per avere il campo generato da tutti i fili, basta usare il principio di
	sovrapposizione e integrare $E_{\text{2 fili}}$ da $0$ a $+\infty$:
	\begin{align*}
	E &= \int_{0}^{+\infty} E_{\text{2 fili}}
	= \int_{0}^{+\infty} \frac{\lambda y}{\pi \varepsilon_0 (x'^2 + y^2)}\\
	&= \int_{0}^{+\infty} \frac{\sigma dx' y}{\pi \varepsilon_0 (x'^2 + y^2)}
	= \frac{\sigma y}{\pi \varepsilon_0} \int_{0}^{+\infty} 
	\frac{dx'}{(x'^2 + y^2)}\\
	&= \frac{\sigma y}{\pi \varepsilon_0} \int_{0}^{+\infty} 
	\frac{dx'}{y^2 \left( \left(\frac{x'}{y}\right)^2 + 1 \right)}\\
	&= \frac{\sigma}{\pi \varepsilon_0} \int_{0}^{+\infty} 
	\frac{\frac{dx'}{y}}{\left( \left(\frac{x'}{y}\right)^2 + 1 \right)}\\
	&= \frac{\sigma}{\pi \varepsilon_0} 
	\left.\arctan\left(\frac{x'}{y}\right)\right\rvert_{0}^{+\infty}
	= \frac{\sigma}{\cancel{\pi} \varepsilon_0} \frac{\cancel{\pi}}{2}
	= \boxed{\frac{\sigma}{2\varepsilon_0}} 
	\end{align*}
	coerentemente con quanto trovato nel caso del disco a raggio infinito.
\end{proof}

\begin{problema}
	Calcolare il campo $\vec{E}$ generato da una lamina di spessore $2a$, che 
	abbia le restanti due dimensioni infinite. La lamina ha una densità 
	volumetrica di carica $\rho = \mathrm{costante}$. 
\end{problema}
\begin{proof}
	Per risolvere il problema, basta dividere la lamina metallica in tante
	``sfoglie'' sottili, di spessore infinitesimo $ds$ e di superficie infinita.
	La densità di carica superficiale delle ``sfoglie'' sarà dunque 
	$\sigma = \rho ds = \mathrm{costante}$, e il campo generato dalla singola
	sfoglia in un punto generico sarà semplicemente:
	$$
	E_{\text{sfoglia}} = \frac{\sigma}{2\varepsilon_0} =
	\frac{\rho ds}{2\varepsilon_0}
	$$
	Per comodità, possiamo posizionare il sistema di riferimento in modo tale che
	lo spessore della lamina sia parallelo al piano $yz$ e l'asse $x$ lo divida
	esattamente a metà. Con questa configurazione $ds = dy$, e utilizzando il
	principio di sovrapposizione è possibile calcolare il campo totale
	integrando il campo della singola sfoglia da $-a$ ad $a$:
	\begin{align*}
	E &= \int_{-a}^{a} E_{\text{sfoglia}} 
	= \int_{-a}^{a} \frac{\rho ds}{2\varepsilon_0}\\
	&= \int_{-a}^{a} \frac{\rho dy}{2\varepsilon_0}
	= \frac{\rho}{2\varepsilon_0} \int_{-a}^{a} dy\\
	&= \frac{\rho}{\cancel{2}\varepsilon_0} \cancel{2}a 
	= \boxed{\frac{\rho a}{\varepsilon_0}}
	\end{align*}
\end{proof}

\end{document}