\documentclass[a4paper,oneside]{article}
\usepackage[utf8]{inputenc}
\usepackage{type1cm}
\usepackage[italian]{babel}
\usepackage[%hypertex,
                 unicode=true,
                 plainpages = false, 
                 pdfpagelabels, 
                 bookmarks=true,
                 bookmarksnumbered=true,
                 bookmarksopen=true,
                 breaklinks=true,
                 backref=false,
                 colorlinks=true,
                 linkcolor = blue,		% Use "blue" if you want to highlight them
                 urlcolor  = blue,
                 citecolor = red,
                 anchorcolor = green,
                 hyperindex = true,
                 linktocpage = true,
                 hyperfigures
]{hyperref}
\usepackage{amsthm}
\renewenvironment{proof}
	{\noindent\textsc{Soluzione:}}
	{\begin{flushright}$\blacksquare$\end{flushright}\vskip 1em}
\usepackage{amsmath}
\usepackage{amsfonts}
\usepackage{amssymb}
\usepackage{cancel}
\usepackage{graphicx}
\usepackage{float}
\usepackage{enumitem}
%\usepackage{comment} % Uncomment for selective compilation
\usepackage{asymptote}
\usepackage{geometry}
\author{A.V.}
\title{Esercizi svolti di Fisica Generale II\\
	\large{\textit{Dalle lezioni di Giuseppe Dalba}}}
\newtheorem{problema}{Problema}
\let\oldhat\hat
\renewcommand{\vec}[1]{\mathbf{#1}}
\renewcommand{\hat}[1]{\widehat{\mathbf{#1}}}
\DeclareMathOperator*{\arsinh}{arsinh}
%\usepackage[square]{natbib}

\begin{document}

\maketitle

\begin{problema}
	Preso un filo sottile carico, di lunghezza $2a$ e 
	distribuzione lineare di carica $\lambda = \mathrm{costante}$, 
	determinare:
	\begin{enumerate}
		\item $\vec{E} = \vec{E}(x,\,y,\,z)$ in un qualsiasi punto 
		che si trovi sull'asse del filo.
		\item Sempre in un generico punto sull'asse, trovare $\vec{E}$ 
		nel limite in cui $a \longrightarrow \infty$.
	\end{enumerate}
\end{problema}
\begin{proof}
	\begin{enumerate}
		\item Da considerazioni di simmetria (vedi Figura \ref{fig:filo_sottile_carico}),
		vale che $\vec{E} \equiv \vec{E}_x$,
		dove $\vec{E}_x$ indica la componente del campo $\vec{E}$ nella direzione $x$.
		\begin{figure}
			\centering
			\begin{asy}
				import graph;
				import cse5;
				texpreamble("\let\oldhat\hat
				\renewcommand{\vec}[1]{\mathbf{#1}}
				\renewcommand{\hat}[1]{\oldhat{\mathbf{#1}}}");
				size(8cm);
				real xmin=-2,xmax=12;
				real ymin=-8,ymax=8;
				xaxis(Label("\small $x$",position=EndPoint,align=SE),
				xmin, xmax, arrow=EndArrow, NoTicks);
				yaxis(Label("\small $y$",position=EndPoint,align=NW),
				ymin, ymax, arrow=EndArrow, NoTicks);
				dot(Label("$P(x,\,y,\,z)$",align=NNE),(10,0));
				draw("$\vec{r}$",(0,0)--(10,0),1bp+black,Arrow(10bp,8));
				draw(Label("$\vec{r}-\vec{r'}$",position=MidPoint,NE),
				(0,4)--(10,0),1bp+black,Arrow(8bp,8));
				draw(Label("$\vec{r'}$",position=MidPoint,W),
				(-0.2,0)--(-0.2,4),1bp+black,Arrow(8bp,8));
				draw((0,-6)--(0,6),solid+1.5bp);
				MarkAngle("\vartheta",black,(0,4),(10,0),(0,0),2,black);
				labely("$a$",6,W);
				labely("$-a$",-6,W);
			\end{asy}
			\caption{Filo sottile carico}
			\label{fig:filo_sottile_carico}
		\end{figure}
		Ora,
		$$
		E_x = k_e \int_{l} \frac{\lambda dl (x-x')}{\left[ (x-x')^2 + (y-y')^2 + 
		(z-z')^2 \right]^{3/2}}
		$$
		Poiché $\vec{r} = (x,\,0,\,0)$, $\vec{r'} = (0,\,y',\,0)$ e $dl \equiv dy'$,
		si ottiene che
		$$
		E_x = k_e \int_{-a}^{a} \frac{\lambda dy' x}{\left[ x^2 + y'^2 \right]^{3/2}}
		= k_e \lambda x \int_{-a}^{a} \frac{dy'}{\left[ x^2 + y'^2 \right]^{3/2}}
		$$
		Usando la semplificazione $y' = x\tan\vartheta, \; dy' = 
		\frac{x}{\cos^2\vartheta}d\vartheta$, abbiamo:
		\begin{align*}
		E_x &= k_e \lambda \cancel{x} \int_{\arctan(-a/x)}^{\arctan(a/x)}
		\frac{\cancel{x} d\vartheta}{\cancel{cos^2\vartheta}} 
		\frac{cos^{\cancel{3}^1}\vartheta}{x^{\cancel{3}^1}}\\
		&= \frac{k_e \lambda}{x} \int_{\arctan(-a/x)}^{\arctan(a/x)}
		\cos\vartheta d\vartheta\\ &= 
		\boxed{\frac{2 k_e \lambda}{x} \sin(\arctan(a/x))}
		\end{align*}
		\item Per $a \longrightarrow \infty$, $\arctan(a/x) \longrightarrow \pi/2$.
		Quindi,
		$$
		E \longrightarrow \frac{2 k_e \lambda}{x} \sin\left(\frac{\pi}{2}\right)
		= \frac{2 k_e \lambda}{x} = \boxed{\frac{\lambda}{2 \pi \varepsilon_0 x}}
		$$
		Notiamo che il campo ha lo stesso andamento riscontrato nel caso di 
		cariche puntiformi! Inoltre, per ragioni di natura pratica, conviene 
		equivalentemente studiare $x \ll a$ invece che $a \longrightarrow \infty$.
 	\end{enumerate}
\end{proof}

\begin{problema}
	Presa una spira sottile carica, di raggio $R$ e 
	distribuzione lineare di carica $\lambda = \mathrm{costante}$, 
	determinare:
	\begin{enumerate}
		\item $\vec{E} = \vec{E}(x,\,y,\,z)$ in un qualsiasi punto 
		che si trovi sull'asse della spira.
		\item $\vec{E} = \vec{E}(x,\,y,\,z)$ come nel caso precedente,
		ma considerando i limiti $x \ll 1$ e $x \gg R$.
		\item $\vec{E} = \vec{E}(x,\,y,\,z)$ in un qualsiasi punto 
		che si trovi sull'asse di un disco avente lo stesso raggio
		e densità di carica superficiale $\sigma = \mathrm{costante}$.
		\item $\vec{E} = \vec{E}(x,\,y,\,z)$ come nel caso precedente,
		ma considerando il limite $R \longrightarrow \infty$ (piano infinito).
	\end{enumerate}
\end{problema}
\begin{proof}
	\begin{enumerate}
		\item La soluzione è identica al problema del filo sottile carico: 
		chiamando $x$ l'asse della spira, $\vec{E} \equiv \vec{E}_x$. Considerando
		che $\vec{r} = (x,\,0,\,0)$ e $\vec{r'} = (0,\,y',\,z')$, il calcolo diventa:
		\begin{align*}
		E_x &= k_e \lambda \int_l \frac{dl(x-x')}{|\vec{r} - \vec{r'}|^3}\\
		&= \frac{k_e \lambda x}{|\vec{r} - \vec{r'}|^3} \int_l dl\\
		&= \frac{k_e \lambda x l}{|\vec{r} - \vec{r'}|^3}\\
		&= \boxed{\frac{k_e \lambda x 2\pi R}{(x^2 + R^2)^{3/2}}}
		\end{align*}
		\item Se $x \ll 1$, $E \longrightarrow 0$. Se $x \gg R$, abbiamo invece
		$$
		E \sim \frac{k_e\cancel{x}\lambda 2 \pi R}{x^{\cancel{3}^2}} =
		\frac{k_e\lambda 2 \pi R}{x^2} =
		\frac{1}{\null_2\cancel{4} \cancel{\pi} \varepsilon_0} 
		\frac{k_e\lambda \cancel{2} \cancel{\pi} R}{x^2} =
		\frac{\lambda R}{2 \varepsilon_0 x^2}
		$$
		Se $Q$ è la carica del filo, $Q = \lambda 2 \pi R$, quindi
		$$
		E \sim \boxed{\frac{k_e Q}{x^2}}
		$$
		Il risultato ci dice che, da lontano, la spira è assimilabile 
		ad una carica puntiforme!
		\item Per calcolare il campo generato da un disco, conviene prima
		calcolare il campo $dE_x$ generato da un anello sottile, di spessore infinitesimo
		$dr'$. La superficie infinitesima dell'anello sarà dunque $dS = dr'dl$
		\footnote{
		Equivalentemente, è possibile definire $dS = 2 \pi r' dr'$, cioè prendere
		una corona circolare infinitesima, e integrare direttamente.		
		}.
		Abbiamo:
		\begin{align*}
		dE_x &= k_e \int_S \frac{\sigma dS (x-x')}{|\vec{r} - \vec{r'}|^3}
		= k_e \int_l \frac{\sigma dr'dl x}{(x^2+r'^2)^{3/2}}\\
		&= \frac{k_e \sigma dr' x}{(x^2+r'^2)^{3/2}} \int_l dl
		= \frac{k_e \sigma dr' x l}{(x^2+r'^2)^{3/2}}\\
		&= \frac{k_e \sigma dr' x 2 \pi r'}{(x^2+r'^2)^{3/2}}
		= \frac{\sigma x r' dr'}{2 \varepsilon_0 (x^2+r'^2)^{3/2}}
		\end{align*}
		Ora, basta integrare su tutta la lunghezza del raggio:
		\begin{align*}
		E_x &= \int_{0}^{R} dE_x = \int_{0}^{R} 
		\frac{\sigma x r' dr'}{2 \varepsilon_0 (x^2+r'^2)^{3/2}}\\
		&= \frac{\sigma x}{2 \varepsilon_0} \frac{1}{2} \int_{0}^{R} 
		\frac{2 r' dr'}{(x^2+r'^2)^{3/2}}\\
		&= \frac{1}{\cancel{2}} \frac{\sigma x}{2 \varepsilon_0}
		\left. \frac{(-\cancel{2})}{\sqrt{x^2+r'^2}} \right\lvert_O^R\\
		&= \boxed{\frac{\sigma}{2 \varepsilon_0} \left( 1 - 
		\frac{x}{\sqrt{x^2+R^2}}\right)}
		\end{align*}
		\item Per $x \ll R$, dal caso precedente segue immediatamente che:
		$$
		\boxed{E \sim \frac{\sigma}{2 \varepsilon_0}}
		$$
		A distanza ravvicinata dal disco, $\vec{E}$ si comporta come un campo costante!
	\end{enumerate}
\end{proof}

\begin{problema}
	Calcolare $\vec{E}$ in un generico punto del piano in Figura
	\ref{fig:distribuzioni_parallele}, dove le distribuzioni superficiali 
	di carica schematizzate sono da considerare di lunghezza e larghezza
	infinite.
	\begin{figure}
			\centering
			\begin{asy}
				import graph;
				import markers;
				texpreamble("\let\oldhat\hat
				\renewcommand{\vec}[1]{\mathbf{#1}}
				\renewcommand{\hat}[1]{\oldhat{\mathbf{#1}}}");
				size(8cm);
				real xmin=-8,xmax=8;
				real ymin=-6,ymax=6;
				xaxis(Label("\small $x$",position=EndPoint,align=SE),
				xmin, xmax, arrow=EndArrow, NoTicks);
				yaxis(Label("\small $y$",position=EndPoint,align=NW),
				ymin, ymax, arrow=EndArrow, NoTicks);
				draw(Label("\small $+\sigma$",position=EndPoint,N,p=black),
				(-3,-6)--(-3,6),invisible,
				CrossIntervalMarker(12,4,size=2,angle=45,bp+black,
				dotframe(invisible)));
				draw(Label("\small $+\sigma$",position=EndPoint,N,p=black),
				(-1,-6)--(-1,6),invisible,
				CrossIntervalMarker(12,4,size=2,angle=45,bp+black,
				dotframe(invisible)));
				draw(Label("\small $+\sigma$",position=EndPoint,N,p=black),
				(1,-6)--(1,6),invisible,
				CrossIntervalMarker(12,4,size=2,angle=45,bp+black,
				dotframe(invisible)));
				draw(Label("\small $+\sigma$",position=EndPoint,N,p=black),
				(3,-6)--(3,6),invisible,
				CrossIntervalMarker(12,4,size=2,angle=45,bp+black,
				dotframe(invisible)));
				draw(Label("\small $a$",align=Center,filltype=UnFill),
				(-2.9,-4)--(-1.1,-4),arrow=Arrows(4bp,8));
				draw(Label("\small $a$",align=Center,filltype=UnFill),
				(-0.9,-4)--(0.9,-4),arrow=Arrows(4bp,8));
				draw(Label("\small $a$",align=Center,filltype=UnFill),
				(1.1,-4)--(2.9,-4),arrow=Arrows(4bp,8));
			\end{asy}
			\caption{Distribuzioni di carica parallele}
			\label{fig:distribuzioni_parallele}
		\end{figure}
\end{problema}
\begin{proof}
	Da considerazioni di natura geometrica, si vede chiaramente che il campo
	assume i valori riportati in Figura \ref{fig:distribuzioni_parallele_soluzione}.
	\begin{figure}
			\centering
			\begin{asy}
				import graph;
				import markers;
				texpreamble("\let\oldhat\hat
				\renewcommand{\vec}[1]{\mathbf{#1}}
				\renewcommand{\hat}[1]{\oldhat{\mathbf{#1}}}");
				size(8cm);
				real xmin=-8,xmax=8;
				real ymin=-6,ymax=6;
				xaxis(Label("\small $x$",position=EndPoint,align=SE),
				xmin, xmax, arrow=EndArrow, NoTicks);
				yaxis(Label("\small $y$",position=EndPoint,align=NW),
				ymin, ymax, arrow=EndArrow, NoTicks);
				draw(Label("\small $+\sigma$",position=EndPoint,N,p=black),
				(-3,-6)--(-3,6),invisible,
				CrossIntervalMarker(12,4,size=2,angle=45,bp+black,
				dotframe(invisible)));
				draw(Label("\small $+\sigma$",position=EndPoint,N,p=black),
				(-1,-6)--(-1,6),invisible,
				CrossIntervalMarker(12,4,size=2,angle=45,bp+black,
				dotframe(invisible)));
				draw(Label("\small $+\sigma$",position=EndPoint,N,p=black),
				(1,-6)--(1,6),invisible,
				CrossIntervalMarker(12,4,size=2,angle=45,bp+black,
				dotframe(invisible)));
				draw(Label("\small $+\sigma$",position=EndPoint,N,p=black),
				(3,-6)--(3,6),invisible,
				CrossIntervalMarker(12,4,size=2,angle=45,bp+black,
				dotframe(invisible)));
				label(rotate(90)*"\small $E = -\frac{2\sigma}{\varepsilon_0}$",
				(-4,0),NoAlign,p=solid+black,filltype=UnFill);
				label(rotate(90)*"\small $E = -\frac{\sigma}{\varepsilon_0}$",
				(-2,0),NoAlign,p=solid+black,filltype=UnFill);
				label(rotate(90)*"\small $E = 0$",
				(0,0),NoAlign,p=solid+black,filltype=UnFill);
				label(rotate(90)*"\small $E = \frac{\sigma}{\varepsilon_0}$",
				(2,0),NoAlign,p=solid+black,filltype=UnFill);
				label(rotate(90)*"\small $E = \frac{2\sigma}{\varepsilon_0}$",
				(4,0),NoAlign,p=solid+black,filltype=UnFill);
			\end{asy}
			\caption{Distribuzioni di carica parallele}
			\label{fig:distribuzioni_parallele_soluzione}
		\end{figure}
\end{proof}

\begin{problema}
	Calcolare il campo $\vec{E}$ generato da un piano infinito carico 
	elettricamente, considerandolo come successione infinita di fili di 
	lunghezza a loro volta infinita. Il piano ha una densità
	superficiale di carica $\sigma = \mathrm{costante}$.
\end{problema}
\begin{proof}
	Iniziamo con delle considerazioni di natura geometrica. Prendiamo il piano
	di cariche in modo tale che sia coincidente col piano $xz$ e dividiamolo
	in tanti fili paralleli all'asse $z$, ognuno di spessore infinitesimo $dx$. 
	Dalla condizione $\sigma = \mathrm{costante}$, otteniamo subito che la 
	densità lineare di carica dei fili $\lambda = \sigma dx = \mathrm{costante}$.
	
	Siano ora $x'$ la distanza di un filo generico dall'asse $z$ e $y$ la distanza 
	di un punto generico $P$ dal piano di cariche. Per semplicità, possiamo prendere 
	$\vec{r'} = (x',\,0,\,0)$ e $\vec{r} = (0,\,y,\,0)$, in modo tale che
	$P$ abbia coordinate $(0,\,y,\,0)$ e $dx = dx'$. Il campo nel punto $P$ generato 
	dal filo in posizione $\vec{r'}$ non sarà, in generale, parallelo all'asse $y$. 
	Avrà una componente $E_x$ diretta lungo l'asse $x$ e una componente
	$E_y$ diretta lungo l'asse $y$. 
	Ma se prendiamo il filo in posizione $-\vec{r'}$, quest'ultimo indurrà in $P$
	un campo che avrà componenti $-E_x$ e $E_y$! Quindi il campo totale generato
	dai due fili sarà diretto lungo l'asse $y$ e avrà modulo\footnote{
	Per il campo generato da un filo infinitamente lungo, 
	riguardare il relativo esercizio.	
	}
	$$
	E_{\text{2 fili}} = 2E_y = \frac{\lambda y}{\pi \varepsilon_0 (x'^2 + y^2)}
	$$
	Ora, per avere il campo generato da tutti i fili, basta usare il principio di
	sovrapposizione e integrare $E_{\text{2 fili}}$ da $0$ a $+\infty$:
	\begin{align*}
	E &= \int_{0}^{+\infty} E_{\text{2 fili}}
	= \int_{0}^{+\infty} \frac{\lambda y}{\pi \varepsilon_0 (x'^2 + y^2)}\\
	&= \int_{0}^{+\infty} \frac{\sigma dx' y}{\pi \varepsilon_0 (x'^2 + y^2)}
	= \frac{\sigma y}{\pi \varepsilon_0} \int_{0}^{+\infty} 
	\frac{dx'}{(x'^2 + y^2)}\\
	&= \frac{\sigma y}{\pi \varepsilon_0} \int_{0}^{+\infty} 
	\frac{dx'}{y^2 \left( \left(\frac{x'}{y}\right)^2 + 1 \right)}\\
	&= \frac{\sigma}{\pi \varepsilon_0} \int_{0}^{+\infty} 
	\frac{\frac{dx'}{y}}{\left( \left(\frac{x'}{y}\right)^2 + 1 \right)}\\
	&= \frac{\sigma}{\pi \varepsilon_0} 
	\left.\arctan\left(\frac{x'}{y}\right)\right\rvert_{0}^{+\infty}
	= \frac{\sigma}{\cancel{\pi} \varepsilon_0} \frac{\cancel{\pi}}{2}
	= \boxed{\frac{\sigma}{2\varepsilon_0}} 
	\end{align*}
	coerentemente con quanto trovato nel caso del disco a raggio infinito.
\end{proof}

\begin{problema}
	Calcolare il campo $\vec{E}$ generato da una lamina di spessore $2a$, che 
	abbia le restanti due dimensioni infinite. La lamina ha una densità 
	volumetrica di carica $\rho = \mathrm{costante}$. 
\end{problema}
\begin{proof}
	Per risolvere il problema, basta dividere la lamina metallica in tante
	``sfoglie'' sottili, di spessore infinitesimo $ds$ e di superficie infinita.
	La densità di carica superficiale delle ``sfoglie'' sarà dunque 
	$\sigma = \rho ds = \mathrm{costante}$, e il campo generato dalla singola
	sfoglia in un punto generico sarà semplicemente:
	$$
	E_{\text{sfoglia}} = \frac{\sigma}{2\varepsilon_0} =
	\frac{\rho ds}{2\varepsilon_0}
	$$
	Per comodità, possiamo posizionare il sistema di riferimento in modo tale che
	lo spessore della lamina sia parallelo al piano $yz$ e il piano $xz$ divida
	la lamina esattamente a metà (vedi Figura \ref{fig:lamina_infinita}). 
	Con questa configurazione $ds = dy$, e utilizzando 
	il principio di sovrapposizione è possibile calcolare il campo totale
	all'esterno della lamina integrando il campo della singola sfoglia da 
	$-a$ ad $a$:
	\begin{align*}
	E_{\text{ext}} &= \int_{-a}^{a} E_{\text{sfoglia}} 
	= \int_{-a}^{a} \frac{\rho ds}{2\varepsilon_0}\\
	&= \int_{-a}^{a} \frac{\rho dy}{2\varepsilon_0}
	= \frac{\rho}{2\varepsilon_0} \int_{-a}^{a} dy\\
	&= \frac{\rho}{\cancel{2}\varepsilon_0} \cancel{2}a 
	= \boxed{\frac{\rho a}{\varepsilon_0}}
	\end{align*}
	Il campo avrà segno positivo nel verso positivo dell'asse $y$, negativo nel 
	verso negativo dell'asse $y$.
	\begin{figure}
			\centering
			\begin{asy}
				import graph;
				import patterns;
				texpreamble("\let\oldhat\hat
				\renewcommand{\vec}[1]{\mathbf{#1}}
				\renewcommand{\hat}[1]{\oldhat{\mathbf{#1}}}");
				size(8cm);
				real xmin=-8,xmax=8;
				real ymin=-6,ymax=6;
				xaxis(Label("\small $y$",position=EndPoint,align=SE),
				xmin, xmax, arrow=EndArrow, NoTicks);
				yaxis(Label("\small $z$",position=EndPoint,align=NW),
				ymin, ymax, arrow=EndArrow, NoTicks);
				draw((-2,-4)--(-2,4));
				draw((2,-4)--(2,4));
				draw(Label("\small $\infty$",align=Center,filltype=UnFill),
				(-2,4)--(2,4),dashed);
				draw(Label("\small $\infty$",align=Center,filltype=UnFill),
				(-2,-4)--(2,-4),dashed);
				labelx("\small $-a$",-2,WSW);
				labelx("\small $+a$",2,ESE);
				picture fillpattern()
				{
				picture tiling;
				draw(tiling,(0,-2)--(0,2));
				draw(tiling,(-2,0)--(2,0));
				draw(tiling,(0,-3)--(0,3),invisible);
				draw(tiling,(-3,0)--(3,0),invisible);
				return tiling;
				}
				add("fillpattern",fillpattern());
				path fillarea = (-2,-4)--(-2,4)--(2,4)--(2,-4)--cycle;
				fill(fillarea,pattern("fillpattern"));
			\end{asy}
			\caption{Lamina di spessore $2a$}
			\label{fig:lamina_infinita}
		\end{figure}
	
	Per calcolare il campo all'interno della lamina si segue la stessa procedura,
	cambiando però gli estremi di integrazione. In un punto generico in posizione
	$y$ rispetto al piano $xz$, si ottiene:
	\begin{align*}
	E_{\text{int}} &= - \int_{y}^{a} E_{\text{sfoglia}} 
	+ \int_{-a}^{y} E_{\text{sfoglia}}\\
	&= - \int_{y}^{a} \frac{\rho dy}{2\varepsilon_0}
	+ \int_{-a}^{y} \frac{\rho dy}{2\varepsilon_0}\\
	&= \frac{\rho}{2\varepsilon_0} \left( - \int_{y}^{a} dy 
	+ \int_{-a}^{y} dy \right)\\
	&= \frac{\rho}{2\varepsilon_0} (\cancel{-a} + y + y + \cancel{a}) 
	= \boxed{\frac{\rho y}{\varepsilon_0}}
	\end{align*}
\end{proof}

\begin{problema}
	Sia data una sfera cava di raggio $R$, carica elettricamente con densità
	superficiale di carica $\sigma = \mathrm{costante}$. Determinare
	$\vec{E}$ in un punto generico dello spazio.
\end{problema}
\begin{proof}
	Sfruttando la simmetria del problema, possiamo ridurre le dimensioni da $3$
	a $1$ per semplificare i calcoli. Distinguiamo due casi.
	\begin{figure}[H]
			\centering
			\begin{asy}
				import markers;
				texpreamble("\let\oldhat\hat
				\renewcommand{\vec}[1]{\mathbf{#1}}
				\renewcommand{\hat}[1]{\oldhat{\mathbf{#1}}}");
				size(8cm);
				draw(circle((0,0),0.9));
				draw(circle((0,0),1),invisible,
				CrossIntervalMarker(20,4,size=2,angle=45,bp+black,
				dotframe(invisible)));
				draw(Label("\small $S$"),circle((0,0),1.5),dashed);
				draw((0,0)--rotate(15)*(4,0),dashed);
				draw(Label("\small $R$"),(0,0)--rotate(-45)*(0.9,0));
				draw(Label("\small $r$"),(0,0)--rotate(45)*(-1.5,0));
				draw(rotate(15)*(1.5,-0.1)--rotate(15)*(1.5,0.1),2bp+black);
				draw(Label("\small $d\vec{S}$",position=EndPoint,align=NNW),
				rotate(15)*(1.5,0)--rotate(15)*(2,0),arrow=Arrow);		
				draw(Label("\small $\vec{E}$",position=EndPoint,align=S),
				rotate(15)*(1.5,0)--rotate(15)*(2.5,0),arrow=Arrow);
				draw(Label("\small $r$",position=EndPoint,align=S),
				(0,0)--(4,0),arrow=Arrow);
			\end{asy}
			\caption{Sfera cava}
			\label{fig:sfera_cava}
		\end{figure}
	\begin{itemize}
		\item \underline{$r \geq R$}\\
		Vogliamo sfruttare il Teorema di Gauss:
		$$
		\oint \vec{E} \cdot d\vec{S} = \frac{Q}{\varepsilon_0}
		$$
		A tale scopo, consideriamo una superficie sferica $S$ di raggio $r \geq R$:
		per la natura del campo, $\vec{E}$ sarà sempre perpendicolare a tale
		superficie. Inoltre, poiché $\vec{E}$ dipende solo dalla distanza dal
		centro (una volta fissata la carica generatrice), il modulo di $\vec{E}$ 
		sarà costante in ogni punto della superficie che stiamo considerando
		(vedi Figura \ref{fig:sfera_cava}).
		
		Possiamo quindi scrivere:
		$$
		\oint \vec{E} \cdot d\vec{S} = \oint E dS
		= E \oint dS = E 4 \pi r^2
		$$
		Indicando poi con $S_f$ la superficie della sfera, dalla definizione
		si ottiene immediatamente che:
		$$
		Q = \int \sigma dS_f = \sigma \int dS_f = \sigma 4 \pi R^2
		$$
		Applicando il Teorema di Gauss, segue che:
		$$
		E 4 \pi r^2 = \frac{Q}{\varepsilon_0}
		\qquad \Longrightarrow \qquad
		E = \frac{1}{4 \pi \varepsilon_0} \frac{Q}{r^2}
		$$
		ossia
		$$
		\boxed{E = k_e \frac{Q}{r^2}}
		$$
		come nel caso di una carica puntiforme! In particolare per $r=R$,
		sfruttando il fatto che $Q = \sigma 4 \pi R^2$,
		$$
		E(R) = \frac{\sigma}{\varepsilon_0}
		$$
		\item \underline{$r < R$}\\
		Dal Teorema di Gauss segue banalmente che
		$$
		\boxed{E = 0}
		$$
	\end{itemize}
\end{proof}

\begin{problema}
	Sia data una sfera di raggio $R$, carica elettricamente con densità
	volumetrica di carica $\rho = \mathrm{costante}$. Determinare
	$\vec{E}$ in un punto generico dello spazio.
\end{problema}
\begin{proof}
	Distinguiamo due casi.
	\begin{itemize}
	\item \underline{$r \geq R$}\\
	\`E come nel caso della sfera cava, basta sfruttare il principio di
	sovrapposizione! Quindi, se $Q = \int_{\tau} \rho(\tau)d\tau$ è la carica
	contenuta nella sfera, si ha semplicemente che
	$$
	\boxed{E = k_e \frac{Q}{r^2}}
	$$
	\item \underline{$r < R$}\\
	Sfruttiamo il Teorema di Gauss:
	$$
	\oint \vec{E} \cdot d\vec{S} = \frac{q}{\varepsilon_0}
	$$	
	Valgono:
	$$
	\oint \vec{E} \cdot d\vec{S} = E 4 \pi r^2
	$$
	$$
	q = \rho \frac{4}{3} \pi r^3
	$$
	Eguagliando le due relazioni sopra, troviamo che
	$$
	\boxed{E = \frac{\rho r}{3 \varepsilon_0}}
	$$
	\end{itemize}
\end{proof}

\begin{problema}
	Considerare di nuovo il problema della lamina infinita di spessore
	$2a$, carica elettricamente con densità volumetrica di carica
	$\rho = \mathrm{costante}$. Trovare $\vec{E}$ usando il Teorema di Gauss.
\end{problema}
\begin{proof}
	Distinguiamo due casi.
	\begin{itemize}
		\item \underline{All'esterno della lamina}\\
		Consideriamo un cilindro di altezza $h \geq 2a$ e raggio di base $r$,
		posizionato in modo che abbia le due basi parallele alle facce della 
		lamina. Calcoliamo il flusso:
		$$
		\oint \vec{E} \cdot d\vec{S} 
		= \cancelto{0}{\int\limits_{\text{sup. later.}} \vec{E} \cdot d\vec{S}}
		+ \int\limits_{\text{base 1}} \vec{E} \cdot d\vec{S} 
		+ \int\limits_{\text{base 2}} \vec{E} \cdot d\vec{S} 
		$$
		Il flusso attraverso la superficie laterale è nullo, perché 
		$\vec{E} \perp d\vec{S}$. D'altra parte, attraverso le due basi,
		$\vec{E} \parallel d\vec{S}$ e quindi $\vec{E} \cdot d\vec{S} = EdS$.
		Il calcolo si semplifica notevolmente:
		$$
		\oint \vec{E} \cdot d\vec{S} 
		= 2\int\limits_{\text{base}} EdS = 2 E \pi r^2
		$$
		D'altra parte, la carica contenuta all'interno del cilindro è
		$Q = \rho \pi r^2 2 a$, quindi applicando il Teorema di Gauss si
		ottiene:
		$$
		\boxed{E = \frac{\rho a}{\varepsilon_0}}
		$$
		\item \underline{All'interno della lamina}\\
		Il calcolo è del tutto analogo a quello del caso precedente. Il cilindro,
		questa volta, avrà un'altezza $h = 2y < 2a$, e la carica contenuta
		all'interno sarà $Q = \rho \pi r^2 2 y$. Applicando il Teorema di Gauss,
		$$
		\boxed{E = \frac{\rho y}{\varepsilon_0}}
		$$
	\end{itemize}
\end{proof}

\begin{problema}
	Due cariche $q$ puntiformi e della stessa grandezza
	sono disposte a distanza $a$ lungo l'asse in
	posizione simmetrica rispetto al piano $y = 0$
	(Figura \ref{fig:dipolo_3d}).
	Si calcoli il campo elettrico in ciascun punto del
	piano nei seguenti casi:
	\begin{enumerate}%[labelindent=\parindent,leftmargin=*,label=\textnormal{\alph*)}]
		\item Le cariche siano dello stesso segno.
		\item Le cariche abbiano polarità opposta.
	\end{enumerate}
	\begin{figure}[H]
			\centering
			\begin{asy}[height=6cm,inline=true,attach=false,viewportwidth=\linewidth]
				import three;
				import graph3;
				import solids;
				currentprojection=obliqueX(-45);
				texpreamble("\let\oldhat\hat
				\renewcommand{\vec}[1]{\mathbf{#1}}
				\renewcommand{\hat}[1]{\oldhat{\mathbf{#1}}}");
				//size(6cm);
				size3(6cm);		//IgnoreAspect
				real xmin=-2,xmax=6;
				real ymin=-4,ymax=6;
				real zmin=-2,zmax=4;
				xaxis3(Label("\small $x$",position=EndPoint,align=SE),
				xmin, xmax, arrow=EndArrow3, NoTicks3);
				yaxis3(Label("\small $y$",position=EndPoint,align=NW),
				ymin, ymax, arrow=EndArrow3, NoTicks3);
				zaxis3(Label("\small $z$",position=EndPoint,align=NW),
				zmin, zmax, arrow=EndArrow3, NoTicks3);
				triple Q1=(0,2,0);
				triple Q2=(0,-2,0);			
				dot(Q1);
				dot(Q2);
				label("\small $q$",Q1,S);
				label("\small $q$",Q2,S);
				draw(Label(shift(0,6)*"\small $\frac{a}{2}$"),
				(0,0.1,0.5)--(0,1.9,0.5),arrow=Arrows3(8bp,8));
				draw(Label(shift(0,6)*"\small $\frac{a}{2}$"),
				(0,-1.9,0.5)--(0,-0.1,0.5),arrow=Arrows3(8bp,8));
			\end{asy}
			\caption{Cariche puntiformi in tre dimensioni}
			\label{fig:dipolo_3d}
		\end{figure}
\end{problema}
\begin{proof}
	Iniziamo col notare che il piano $y=0$ corrisponde all'``asse'' 
	(bidimensionale) del
	segmento che congiunge le due cariche; inoltre, la forza elettrica
	è una forza centrale, quindi possiamo ridurre il problema da 3 a 2
	dimensioni. Detto in altro modo, si tratta di studiare l'andamento
	del campo sull'asse del segmento che congiunge due cariche puntiformi.
	
	Per comodità, scegliamo come asse proprio l'asse $x$; il generico punto
	$(x,\,0,\,0)$ avrà dunque distanza 
	$\sqrt{x^2 + \left(\frac{a}{2}\right)^2}$
	da ognuna delle due cariche. Il modulo del campo $\vec{E}$ nel punto preso 
	in considerazione sarà chiaramente
	$$
	E = k_e \frac{q}{x^2 + \left(\frac{a}{2}\right)^2}
	$$
	e la direzione di $\vec{E}$ formerà con l'asse $x$ un angolo $\vartheta$
	tale che:
	$$
	\cos\vartheta = \frac{x}{\sqrt{x^2 + \left(\frac{a}{2}\right)^2}};
	\qquad
	\sin\vartheta = \frac{a/2}{\sqrt{x^2 + \left(\frac{a}{2}\right)^2}}.
	$$
	Fatte queste premesse, possiamo ora distinguere i due casi.
	\begin{enumerate}
		\item\underline{Cariche dello stesso segno}\\
		Supponiamo, per semplicità, che le due cariche siano positive
		(se fossero entrambe negative il campo elettrico differirebbe solo
		per il verso). La situazione è schematizzata in Figura 
		\ref{fig:dipolo_2d_pos}.
		\begin{figure}[H]
			\centering
			\begin{asy}
				import cse5;
				texpreamble("\let\oldhat\hat
				\renewcommand{\vec}[1]{\mathbf{#1}}
				\renewcommand{\hat}[1]{\oldhat{\mathbf{#1}}}");
				size(6cm);
				draw(Label("\small $\frac{a}{2}$"),(-2,0)--(0,0));
				draw((0,0)--(2,0));
				filldraw(circle((-2,0),0.2));
				filldraw(circle((2,0),0.2));
				label("\small $q_1^+$",(-2,0),2SW);
				label("\small $q_2^+$",(2,0),2SE);
				draw(Label("\small $x$"),(0,0)--(0,4));
				MarkAngle("\vartheta",black,(-2,0),(0,4),(0,0),1,black);
				draw((-2,0)--(0,4)--(2,0));
				draw(Label("\small $\vec{E_1}$",position=EndPoint),(0,4)--(1,6),
				linewidth(1bp),Arrow(10bp,12));
				draw(Label("\small $\vec{E_2}$",position=EndPoint),(0,4)--(-1,6),
				linewidth(1bp),Arrow(10bp,12));
				draw(Label("\small $\vec{E_{\mathrm{tot}}}$",position=EndPoint),
				(0,4)--(0,8),linewidth(1bp),Arrow(10bp,12));				
			\end{asy}
			\caption{Due cariche positive}
			\label{fig:dipolo_2d_pos}
		\end{figure}
		Facendo il calcolo,
		\begin{align*}
		E_{\mathrm{tot}} &= 2E\cos\vartheta
		= 2 k_e \frac{q}{x^2 + \left(\frac{a}{2}\right)^2}
		\frac{x}{\sqrt{x^2 + \left(\frac{a}{2}\right)^2}}\\
		&= \frac{2 k_e q x}
		{\left( x^2 + \left(\frac{a}{2}\right)^2 \right)^{3/2}}
		= \boxed{\frac{qx}
		{2 \pi \varepsilon_0 \left( x^2 + \left(\frac{a}{2}\right)^2 \right)^{3/2}}}
		\end{align*}
		\item\underline{Cariche di segno opposto}\\
		Supponiamo, per semplicità, che la carica di sinistra in Figura
		\ref{fig:dipolo_2d_pos_neg} sia positiva e quella a destra negativa
		(se fosse al contrario il campo elettrico differirebbe solo
		per il verso).
		\begin{figure}[H]
			\centering
			\begin{asy}
				import cse5;
				texpreamble("\let\oldhat\hat
				\renewcommand{\vec}[1]{\mathbf{#1}}
				\renewcommand{\hat}[1]{\oldhat{\mathbf{#1}}}");
				size(6cm);
				draw(Label("\small $\frac{a}{2}$"),(-2,0)--(0,0));
				draw((0,0)--(2,0));
				filldraw(circle((-2,0),0.2));
				filldraw(circle((2,0),0.2));
				label("\small $q_1^+$",(-2,0),2SW);
				label("\small $q_2^-$",(2,0),2SE);
				draw(Label("\small $x$"),(0,0)--(0,4));
				MarkAngle("\vartheta",black,(-2,0),(0,4),(0,0),1,black);
				draw((-2,0)--(0,4)--(2,0));
				draw(Label("\small $\vec{E_1}$",position=EndPoint,NE),(0,4)--(1,6),
				linewidth(1bp),Arrow(10bp,12));
				draw(Label("\small $\vec{E_2}$",position=EndPoint,SE),(0,4)--(1,2),
				linewidth(1bp),Arrow(10bp,12));
				draw(Label("\small $\vec{E_{\mathrm{tot}}}$",position=EndPoint),
				(0,4)--(2,4),linewidth(1bp),Arrow(10bp,12));				
			\end{asy}
			\caption{Due cariche di segno opposto}
			\label{fig:dipolo_2d_pos_neg}
		\end{figure}
		Facendo il calcolo,
		\begin{align*}
		E_{\mathrm{tot}} &= 2E\sin\vartheta
		= 2 k_e \frac{q}{x^2 + \left(\frac{a}{2}\right)^2}
		\frac{a/2}{\sqrt{x^2 + \left(\frac{a}{2}\right)^2}}\\
		&= \frac{k_e q a}
		{\left( x^2 + \left(\frac{a}{2}\right)^2 \right)^{3/2}}
		= \boxed{\frac{qa}
		{4 \pi \varepsilon_0 \left( x^2 + \left(\frac{a}{2}\right)^2 \right)^{3/2}}}
		\end{align*}
	\end{enumerate}
\end{proof}

\begin{problema}
	Un cilindro di raggio $R$ e lunghezza $2L$ è disposto verticalmente 
	con il proprio asse coincidente con l'asse $z$ ed è centrato 
	rispetto al piano $z=0$. Sulla parete sottile del cilindro è distribuita
	uniformemente una carica di densità superficiale $\sigma$.
	\begin{enumerate}
		\item Si calcoli il campo elettrostatico lungo l'asse del cilindro.
		\item Si valuti il campo per $L \longrightarrow 0$ e per 
		$L \longrightarrow \infty$.
	\end{enumerate}
	\begin{figure}%[H]
			\centering
			\begin{asy}[height=6cm,inline=true,attach=false,viewportwidth=\linewidth]
				import three;
				import graph3;
				import solids;
				//currentprojection=obliqueX(-45);
				currentprojection=orthographic(2,2,1);
				texpreamble("\let\oldhat\hat
				\renewcommand{\vec}[1]{\mathbf{#1}}
				\renewcommand{\hat}[1]{\oldhat{\mathbf{#1}}}");
				size3(8cm);
				real xmin=-4,xmax=5;
				real ymin=-4,ymax=5;
				real zmin=-6,zmax=6;
				xaxis3(Label("\small $x$",position=EndPoint,align=SE),
				xmin, xmax, arrow=EndArrow3, NoTicks3);
				yaxis3(Label("\small $y$",position=EndPoint,align=NW),
				ymin, ymax, arrow=EndArrow3, NoTicks3);
				zaxis3(Label("\small $z$",position=EndPoint,align=NW),
				zmin, zmax, arrow=EndArrow3, NoTicks3);
				revolution lat_surf=cylinder((0,0,-4), 2, 8, Z);
				//path3 base_upper=circle((0,0,4), 2);
				//path3 base_lower=circle((0,0,4), 2);
				draw(lat_surf,black);
				draw(surface(lat_surf),lightgrey+opacity(0.75));
				//draw(surface(base_upper),lightgrey+opacity(0.75));
				//draw(surface(base_lower),lightgrey+opacity(0.75));
				draw(Label("$R$",position=EndPoint,align=N),
				(0,0,4)--(-sqrt(2),sqrt(2),4));
				dot((0,0,4));
				labelz("$L$",(0,0,4),NW);
				dot((0,0,-4));
				labelz("$-L$",(0,0,-4),SW);
			\end{asy}
			\caption{Cilindro cavo verticale}
			\label{fig:cilindro_cavo}
		\end{figure}	
\end{problema}
\begin{proof}
	Considerando $h$ come la distanza fra il piano identificato da $z=0$ e il punto 
	generico $(0,\,0,\,z)$ sulla quale calcoliamo il campo elettrico, ho che ogni 
	superficie $dS$ dà un contributo pari a:
	$$
	|d\vec{E}|=k_e \frac{\sigma (z-h) dS}{|(\vec{r}-\vec{r'})|^3}
	$$
	dove $dS=2\pi R dh$ e $|(\vec{r}-\vec{r'})|=\sqrt{R^2+(z-h)^2}$.
	
	Integrando fra $-L$ e $L$ ottengo:
	$$
	|\vec{E}|=\int_{-l}^{L} k_e \frac{\sigma (z-h) dS}{|(\vec{r}-\vec{r'})|^3}=
	k_e \int_{-L}^{L} \frac{\sigma (z-h) 2\pi R }
	{\left(\sqrt{R^2+(z-h)^2}\right)^3} dh
	$$
	Posso risolvere l'integrale effettuando la sostituzione $s=z-h$ ($ds=-dh$); 
	l'integrale si riduce a:
	$$
	|\vec{E}| = - 2 k_e \sigma \pi R \int \frac{s}{(\sqrt{R^2+s^2})^3} ds=
	2 k_e \sigma \pi R \left[ \frac{1}{\sqrt{R^2+(z-h)^2}} \right]_{-L}^{L}
	$$
	da cui il risultato:
	$$
	\boxed{
	|\vec{E}|=
	2 k_e \sigma \pi R \left[ \frac{1}{\sqrt{R^2+(z-L)^2}} - 
	\frac{1}{\sqrt{R^2+(z+L)^2}} \right]
	}
	$$
	
	Per $L \longrightarrow 0$, il campo elettrico è 0, così come per 
	$L \longrightarrow \infty$.
\end{proof}

\begin{problema}
	Una carica di densità volumetrica è distribuita uniformemente in un 
	volume cilindrico di raggio $R$ e lunghezza infinita. Si calcoli 
	il campo elettrostatico ovunque nello spazio.
	\begin{figure}%[H]
			\centering
			\begin{asy}[height=6cm,inline=true,attach=false,viewportwidth=\linewidth]
				import three;
				import graph3;
				import solids;
				//currentprojection=obliqueX(-45);
				currentprojection=orthographic(2,2,1);
				texpreamble("\let\oldhat\hat
				\renewcommand{\vec}[1]{\mathbf{#1}}
				\renewcommand{\hat}[1]{\oldhat{\mathbf{#1}}}");
				size3(8cm);
				real xmin=-4,xmax=5;
				real ymin=-4,ymax=5;
				real zmin=-6,zmax=6;
				xaxis3(Label("\small $x$",position=EndPoint,align=SE),
				xmin, xmax, arrow=EndArrow3, NoTicks3);
				yaxis3(Label("\small $y$",position=EndPoint,align=NW),
				ymin, ymax, arrow=EndArrow3, NoTicks3);
				zaxis3(Label("\small $z$",position=EndPoint,align=NW),
				zmin, zmax, arrow=EndArrow3, NoTicks3);
				revolution lat_surf=cylinder((0,0,-4), 2, 8, Z);
				path3 base_upper=circle((0,0,4), 2);
				path3 base_lower=circle((0,0,4), 2);
				draw(lat_surf,black);
				draw(surface(lat_surf),lightgrey+opacity(0.75));
				draw(surface(base_upper),lightgrey+opacity(0.75));
				draw(surface(base_lower),lightgrey+opacity(0.75));
				draw(Label("$R$",position=EndPoint,align=N),
				(0,0,4)--(-sqrt(2),sqrt(2),4));
				//dot((0,0,4));
				labelz("$+\infty$",(0,0,4),NW);
				//dot((0,0,-4));
				labelz("$-\infty$",(0,0,-4),SW);
			\end{asy}
			\caption{Volume cilindrico di lunghezza infinita}
			\label{fig:cilindro_pieno_infinito}
		\end{figure}	
\end{problema}
\begin{proof}
	Per comodità pongo il generico punto in cui voglio calcolare il campo 
	elettrico sull'asse $y$. Per simmetria (dato che ogni elemento 
	infinitesimo di volume ha un corrispettivo simmetricamente opposto sul 
	cilindro) le componenti diverse da quelle sull'asse delle $y$ si annullano: 
	il campo va dunque calcolato sull'asse $y$.
	
	Il contributo infinitesimo di ogni coppia di volumi infinitesimi è dato da:
	$$
	|d\vec{E}|=2 k_e \frac{\rho dV y}{|(\vec{r}-\vec{r'})|^3}
	$$
	
	Con considerazioni geometriche sul disegno (pongo $y$ la distanza fra il
	volume infinitesimo e il punto su cui voglio calcolare il campo, ed $h$ 
	la distanza fra il piano $z=0$ e il volume infinitesimo) ottengo le seguenti 
	relazioni: $h = y \tan\vartheta$ e $\cos\vartheta |\vec{r}-\vec{r'}|=y$, da cui 
	ricavo $dh=y \frac{1}{cos^2 \vartheta} d\vartheta$. Sostituendo:
	$$
	|d\vec{E}|=k_e \frac{2 \rho \pi R^2 y^2 \frac{1}{\cos^2\vartheta} d\vartheta}
	{\frac{y^3}{\cos^3\vartheta}}=k_e \frac{2 \rho R^2 \cos\vartheta}{y} d\vartheta
	$$
	Integrando fra $0$ e $\pi/2$ ottengo il risultato:
	\begin{equation}
	\label{sol_pr3}
	E_y=k_e \frac{2 \rho \pi R^2}{y}=
	\boxed{
	\frac{\rho R^2}{2 y \varepsilon_0}
	}
	\end{equation}
	
	Lo stesso risultato poteva essere ottenuto sfruttando il teorema di Gauss 
	e considerando come superficie un cilindro di raggio $y$ con l'asse 
	coincidente a quella del nostro cilindro. Dato che il campo elettrico è 
	diretto verso l'asse $y$ per le considerazioni di cui sopra, si può considerare 
	solo la superficie esterna del cilindro. Vale dunque (se consideriamo i cilindri 
	della stessa lunghezza $L$):
	\begin{equation}
	\label{pr3_gauss}
	E 2 \pi y L = \frac{\pi R^2 L \rho}{\varepsilon_0}
	\end{equation}
	Il campo elettrico non dipende dunque da $L$ ed ottengo lo stesso 
	risultato (\ref{sol_pr3}).
	
	Con il teorema di Gauss è possibile calcolare facilmente anche il campo elettrico 
	nel caso in cui $y<R$, sostituendo $y$ a $R$ (per il teorema di Gauss consideriamo 
	solo le cariche interne), ottenendo:
	$$
	\boxed{
	E_y=\frac{\rho y}{2 \varepsilon_0}
	}
	$$
\end{proof}

\begin{problema}
	Una linea di trasmissione è costituita da un cavo sottile
	rettilineo ed infinitamente lungo su cui è distribuita
	uniformemente una carica di densità $\lambda = 10^{-7} \; \mathrm{C/m}$.
	La linea è disposta parallelamente al suolo ad una distanza
	$D = 10 \; \mathrm{m}$.
	Trascurando l'influenza del terreno si calcoli:
	\begin{enumerate}
		\item Il campo elettrico ovunque nello spazio.
		\item La grandezza del campo elettrico sul terreno giusto
		al disotto della linea.
	\end{enumerate}
	Sotto la prima linea ne viene aggiunta, parallelamente ad essa, 
	una seconda a distanza $d = 2 \; \mathrm{m}$ su cui è
	distribuita uniformemente una carica di densità lineare uguale ed opposta, 
	pari cioè a $-10^{-7} \; \mathrm{C/m}$. Si calcoli:
	\begin{enumerate}[start=3]
		\item Il campo elettrico in un generico punto dello spazio.
		\item Il campo elettrico al livello del suolo immediatamente 
		al disotto delle linee.
	\end{enumerate}
\end{problema}
\begin{proof}
	\begin{enumerate}
		\item Avevamo già calcolato il campo generato da un filo infinito.
		Ricordiamo che, a distanza $x$ dall'asse del filo, il campo
		è perpendicolare al filo stesso e in vale:
		$$
		\boxed{
		\vec{E} = \frac{\lambda}{2\pi\varepsilon_0x}\hat{x}
		}
		$$
		\item Basta porre $x = D$ nella formula precedente per ottenere:
		$$
		\vec{E_D} = \frac{\lambda}{2\pi\varepsilon_0D}\hat{x} \simeq
		\boxed{179.95 \; \mathrm{N/C}}
		$$
		\item Preso un generico punto P, indichiamo con $x^+$ la sua distanza 
		dall'asse del filo con densità di
		carica $\lambda^+$ positiva, e con $x^-$ la sua distanza dall'asse del filo
		con densità di carica $\lambda^-$ negativa. Poniamo inoltre 
		$\lambda = |\lambda^+| = |\lambda^-|$.
		
		Dal principio di sovrapposizione e utilizzando la formula al primo punto,
		otteniamo:
		$$
		\vec{E} =
		\frac{\lambda}{2\pi\varepsilon_0x^+}\hat{x}^+ -
		\frac{\lambda}{2\pi\varepsilon_0x^-}\hat{x}^- =
		\boxed{
		\frac{\lambda}{2\pi\varepsilon_0}\left(
		\frac{\hat{x}^+}{x^+} - \frac{\hat{x}^-}{x^-}
		\right)
		}
		$$
		Il segno del campo può essere (arbitrariamente) riferito al filo positivo,
		per cui un segno ``$+$'' nel risultato indicherà un campo \emph{uscente}
		dal filo positivo, mentre un ``$-$'' un campo \emph{entrante}.
		\item Basta sostituire $x^+ = D$, $x^- = D-d$ e 
		$\hat{x}^+ = \hat{x}^-$ nella formula precedente per ottenere:
		$$
		\vec{E_D} = - \frac{\lambda d}{2\pi\varepsilon_0D(D-d)}\hat{x}^+
		\simeq \boxed{- 44.94 \; \mathrm{N/C}}
		$$
		dove il segno ``$-$'' indica che il campo è diretto verso il filo
		di densità di carica positiva.
	\end{enumerate}
\end{proof}

\begin{problema}
	Una carica è distribuita su un filo sottile infinitamente lungo ripiegato 
	su se stesso come in Figura \ref{fig:filo_carico_ricurvo}. I tratti rettilinei sono 
	paralleli tra loro. La densità di carica lineare è uguale a $\lambda$; 
	il raggio di curvatura del filo è $R$.
	
	Calcolare il campo elettrico lungo l'asse $z$ passante per il centro della 
	circonferenza di raggio R.
\end{problema}
\begin{proof}
	\footnote{
	La soluzione qui proposta non calcola il campo \emph{lungo} l'asse $z$,
	ma \emph{in ogni punto} dell'asse $z$, ed è quindi più generale
	rispetto a quanto richiesto dal problema.}
	Per il principio di sovrapposizione posso considerare il campo elettrico come 
	la somma di tre contributi: due dati dai fili infiniti paralleli e uno dato 
	dalla semicirconferenza.
	\begin{figure}%[H]
			\centering
			\begin{asy}[height=6cm,inline=true,attach=false,viewportwidth=\linewidth]
				import three;
				import graph3;
				import solids;
				currentprojection=orthographic(1,-2,1);
				texpreamble("\let\oldhat\hat
				\renewcommand{\vec}[1]{\mathbf{#1}}
				\renewcommand{\hat}[1]{\oldhat{\mathbf{#1}}}");
				size3(8cm);
				real xmin=-7,xmax=6;
				real ymin=-7,ymax=6;
				real zmin=-2,zmax=6;
				xaxis3(Label("\small $x$",position=EndPoint,align=SE),
				xmin, xmax, arrow=EndArrow3, NoTicks3);
				yaxis3(Label("\small $y$",position=EndPoint,align=NW),
				ymin, ymax, arrow=EndArrow3, NoTicks3);
				zaxis3(Label("\small $z$",position=EndPoint,align=NW),
				zmin, zmax, arrow=EndArrow3, NoTicks3);
				path3 semicirc = (4,0,0)..(0,4,0)..(-4,0,0);
				path3 line1 = (-4,0,0)--(-4,-6,0);
				path3 line1infty = (-4,-6,0)--(-4,-7,0);
				path3 line2 = (4,0,0)--(4,-6,0);
				path3 line2infty = (4,-6,0)--(4,-7,0);
				draw(semicirc, linewidth(2bp)+solid);
				draw(line1, linewidth(2bp)+solid);
				draw(line1infty, linewidth(2bp)+dotted);
				draw(line2, linewidth(2bp)+solid);
				draw(line2infty, linewidth(2bp)+dotted);
				draw(Label("\small $R$",align=N),(0,0,0)--(4*sqrt(3)/2,4*1/2,0));
				draw(Label("\small $\vec{r'}$",position=MidPoint,align=N),
				(0,0,0)--(-4*sqrt(3)/2,4*1/2,0),linewidth(1bp),Arrow3(10bp,12));
				draw(Label("\small $\vec{r-r'}$",position=MidPoint,align=NW),
				(-4*sqrt(3)/2,4*1/2,0)--(0,0,4),linewidth(1bp),Arrow3(10bp,12));
				draw(Label("\small $\vec{r}$",position=MidPoint,align=E),
				(0,0,0)--(0,0,4),linewidth(1bp),Arrow3(10bp,12));
				draw((0,0,0.5)--(-sqrt(3)/4,1/4,0.5)--(-sqrt(3)/4,1/4,0));
				draw(Label("\small $\psi$",align=WNW),(-1,0,0)..
				(-(sqrt(6)+sqrt(2))/4,(sqrt(6)-sqrt(2))/4,0)..(-sqrt(3)/2,1/2,0));
				draw(Label("\small $\vartheta$",align=SW),(0,0,3)..
				(-(sqrt(2-sqrt(2)))/2*sqrt(3)/2,(sqrt(2-sqrt(2)))/2*1/2,4-(sqrt(2+sqrt(2)))/2)..
				(-sqrt(2)/2*sqrt(3)/2,sqrt(2)/2*1/2,4-sqrt(2)/2));
				dot(Label("\small $P$",align=SE),(0,0,4));
				draw(Label("\small $d\vec{E}$",position=EndPoint,align=SE),
				(0,0,4)--(4*sqrt(3)/4,-1,6),linewidth(1bp),Arrow3(10bp,12));
			\end{asy}
			\caption{Filo carico ricurvo}
			\label{fig:filo_carico_ricurvo}
		\end{figure}	
	
	Calcoliamo il contributo al campo elettrico dato dalla semicirconferenza. 
	Si nota che la componente sull'asse $x$ del campo è nulla per simmetria. 
	Calcoliamo dunque il campo sugli assi $y$ e $z$.
	Vale:
	$$
	d\vec{E}=
	k_e \frac{\lambda (\vec{r}-\vec{r'}) dl}{|\vec{r}-\vec{r'}|^3}
	$$
	
	Consideriamo i vettori posizione $\vec{r}=(0,0,z)$ (vettore che identifica 
	il generico punto dell'asse $z$ nel quale sto calcolando il campo) e 
	$\vec{r'}=(x',y',0)$, $R$ il raggio della semicirconferenza e $\psi \in [0,\,\pi]$ 
	l'angolo sul piano $z=0$ compreso fra il vettore $\vec{r'}$ e l'asse $x$. 
	Il contributo infinitesimo al campo elettrico sull'asse $y$ è dato da:
	$$
	d\vec{E_y}=
	k_e \frac{\lambda (-R \sin \psi) dl}{(R^2+z^2)^{3/2}} \hat{y}=
	k_e \frac{\lambda (-R \sin \psi) R d\psi}{(R^2+z^2)^{3/2}}  \hat{y},
	$$
	da cui ho:
	$$
	\vec{E_y}=
	k_e \int_0^{\pi} \frac{\lambda (-R \sin \psi) R d\psi}{(R^2+z^2)^{3/2}} \hat{y} = 
	- 2 k_e \frac{\lambda R^2 }{(R^2+z^2)^{3/2}} \hat{y}.
	$$
	Il contributo al campo elettrico sull'asse $z$ è invece dato da (considerando 
	$\vartheta$ l'angolo costante fra l'asse $z$ e il vettore $d\vec{E}$):
	$$
	d\vec{E_z}=
	k_e \frac{\lambda R d\psi \cos \vartheta}{(R^2+z^2)} \hat{z}=
	k_e \frac{\lambda R \psi \frac{z}{\sqrt{R^2+z^2}}}{(R^2+z^2)} \hat{z}=
	k_e\frac{\lambda R z d\psi}{(R^2+z^2)^{3/2}} \hat{z}
	$$
	e integrando lungo tutta la semicirconferenza:
	$$
	\vec{E_y}=
	k_e \int_0^{\pi} \frac{\lambda R z d\psi}{(R^2+z^2)^{3/2}} \hat{z} =
	k_e \frac{\lambda R z \pi}{(R^2+z^2)^{3/2}} \hat{z}
	$$
	Il filo di lunghezza infinitesima contribuisce al campo, per le stesse 
	considerazioni di simmetria di cui sopra, sull'asse $y$ e $z$.
	Per semplificare possiamo trattare il problema come se si svolgesse su un unico 
	piano senza mancare di generalità: infatti, il campo elettrico generato da un 
	filo disposto su una semiretta di lunghezza infinita, sui punti disposti lungo 
	la retta perpendicolare al filo passante per la sua estremità, dipende solo dalla 
	distanza.
	Consideriamo tale distanza $\sqrt{R^2+z^2}$, abbiamo che 
	(si veda esercizio I.22 \cite{mencuccini}, pag. 72: le formule utilizzate in seguito 
	sono state ricavate facendo tendere ad infinito la lunghezza del filo) la 
	componente $z$ (componente $y$ sull'esercizio del \cite{mencuccini} moltiplicata 
	per $\cos \vartheta$) è:
	$$
	\vec{E_z}=
	2 k_e \frac{\lambda}{\sqrt{R^2+z^2}} \cos \vartheta \hat{z}=
	2 k_e \frac{\lambda z}{R^2+z^2} \hat{z}
	$$
	La componente $y$ è invece:
	$$
	\vec{E_y}=
	2 k_e \frac{\lambda}{\sqrt{R^2+z^2}} \hat{y}
	$$
	La soluzione è dunque:
	$$
	\boxed{
	\vec{E}=
	\left(2 k_e \frac{\lambda}{\sqrt{R^2+z^2}}- 
	2 k_e \frac{\lambda R^2 }{(R^2+z^2)^{3/2}}\right)\hat{y}+
	\left(k_e\frac{\lambda R z \pi}{(R^2+z^2)^{3/2}}+
	2 k_e \frac{\lambda z}{R^2+z^2}\right)\hat{z}
	}
	$$
	
	Per completezza, riportiamo come ottenere il campo generato dai due fili per
	integrazione diretta. Usando la simmetria del sistema, il campo totale
	generato dai due fili sarà il doppio di quello generato da un filo.
	Detta $l \in [0,\,+\infty]$ la distanza lungo il filo a partire dall'asse $x$,
	la componente lungo $z$ del campo è
	\begin{align*}
	d\vec{E_z} &= 2 k_e \frac{\lambda dl}{l^2+R^2+z^2} \cos\vartheta \hat{z}\\
	&= 2 k_e \frac{\lambda dl}{l^2+R^2+z^2} \frac{z}{\sqrt{l^2+R^2+z^2}} \hat{z}=
	2 k_e \frac{\lambda z dl}{(l^2+R^2+z^2)^{3/2}} \hat{z}
	\end{align*}
	Integrando su tutta la lunghezza,
	\begin{align*}
	\vec{E_z} &= 
	\int_{0}^{+\infty} d\vec{E_z}\\
	&= \int_{0}^{+\infty} 2 k_e \frac{\lambda z dl}{(l^2+R^2+z^2)^{3/2}} \hat{z}\\
	&= 2k_e\lambda z \int_{0}^{+\infty} \frac{dl}{(l^2+R^2+z^2)^{3/2}} \hat{z}
	\end{align*}
	L'integrale può essere risolto con la sostituzione $l=\sqrt{z^2+R^2}\tan u$,
	$dl = \frac{\sqrt{z^2+R^2}}{\cos^2u}du$, $\frac{1}{1+\tan^2u} = \cos^2u$, con la
	quale il calcolo si semplifica in:
	$$
	\vec{E_z} = 2k_e\frac{\lambda z}{(R^2+z^2)} \int_{0}^{\pi/2} \cos u du \hat{z}=
	\boxed{2k_e\frac{\lambda z}{(R^2+z^2)}\hat{z}}.
	$$
	Similmente per l'altra componente.
\end{proof}

\begin{problema}
	Una carica è distribuita uniformemente lungo un filo sottile di lunghezza
	$L$ con densità uniforme $\lambda$. Inizialmente il filo è disposto a distanza
	$d$ da una superficie piana infinitamente grande, carica uniformemente con
	densità di carica $\sigma$. Calcolare il lavoro richiesto per ruotare il filo di
	$90^{\circ}$ come in Figura \ref{fig:spostamento_filo}.
	\begin{figure}
		\centering
		\begin{asy}
			import graph;
			texpreamble("\let\oldhat\hat
			\renewcommand{\vec}[1]{\mathbf{#1}}
			\renewcommand{\hat}[1]{\oldhat{\mathbf{#1}}}");
			size(8cm);
			pen shortdashed=linetype(new real[] {4,4});
			draw(Label("$\sigma$",position=MidPoint,align=N),(0,0)--(2,0),
			2bp+black+shortdashed);
			draw((2,0)--(10,0),2bp+black);
			draw((10,0)--(12,0),2bp+black+shortdashed);
			draw(Label("$\lambda$",position=MidPoint,align=N),(4,2)--(8,2),2bp+black);
			draw((8,2)--(8,6),dashed);
			draw((7,2)..rotate(-45,(8,2))*(7,2)..rotate(-90,(8,2))*(7,2),Arrow);
			draw(Label("$L$",align=Center,filltype=UnFill),(4,1)--(8,1),arrow=Arrows,bar=Bars);
			draw(Label("$d$",align=Center,filltype=UnFill),(9,0)--(9,2),arrow=Arrows,bar=Bars);
		\end{asy}
		\caption{Spostamento di un filo carico}
		\label{fig:spostamento_filo}
	\end{figure}
\end{problema}
\begin{proof}
	Iniziamo col dire che il lavoro \emph{richiesto} è pari al lavoro delle forze
	del campo cambiato di segno. Nel nostro caso,
	$$
	W_R = -W_E
	$$
	dove $W_R$ indica il lavoro richiesto e $W_E$ il lavoro fatto falla forza
	elettrica.
	
	Consideriamo ora un pezzettino di lunghezza infinitesima $dl$ sul filo,
	a distanza $l$ dall'estremo di rotazione, e indichiamo con $dh$ 
	lo spostamento infinitesimo lungo la direzione di 
	$\vec{E}$. Poiché $\vec{E}$ è conservativo, possiamo pensare di sostituire l'arco 
	di circonferenza compiuto dal pezzettino con una spezzata che procede in direzione
	di $\vec{E}$ per il primo tratto, e poi continua orizzontalmente fino al punto di
	arrivo. Il lavoro compiuto nel tratto orizzontale è nullo, perché 
	$\vec{E} \perp \vec{ds}$;
	quindi il lavoro richiesto per ruotare il suddetto pezzettino di un angolo
	pari a $\pi/2$ sarà:
	$$
	dW_R = -\int (\lambda dl)\vec{E} \cdot \vec{ds} =
	-\int_{0}^{l} (\lambda dl)Edh = -\lambda dlE \int_{0}^{l} dh = 
	-\lambda dlEl
	$$
	dove nel calcolo abbiamo usato il fatto che, per un piano infinito, $E$ è costante
	e non dipende dalla distanza.
	
	Per trovare il lavoro totale, basta ora integrare su tutta la lunghezza $L$
	del filo:
	$$
	W_R = \int dW_R = \int_{0}^{L} -\lambda E ldl =
	-\lambda E \int_{0}^{L} l dl = 
	-\lambda E \left.\frac{l^2}{2}\right\rvert_{0}^{L} = -\lambda E \frac{L^2}{2}
	$$
	
	Dal Teorema di Gauss, sappiamo che per una piastra infinita carica uniformemente
	vale:
	$$
	E = \frac{\sigma}{2\varepsilon_0}
	$$
	Sostituendo l'espressione di $E$ in quella di $W_R$, otteniamo infine:
	$$
	\boxed{W_R = -\frac{\lambda \sigma L^2}{4\varepsilon_0}}
	$$
	Il segno ``$-$'' nel risultato è indicativo del fatto che il sistema non
	ha bisogno di un lavoro esterno per posizionare il filo in quella posizione;
	essendo infatti le due distribuzioni con lo stesso segno, la piastra e il filo
	tendono a respingersi, e il filo si posiziona spontaneamente in verticale.
	
	\underline{Soluzione alternativa.} Il lavoro richiesto può essere anche pensato come
	il lavoro necessario a spostare il ``centro di carica'' nella posizione desiderata.
	La carica del centro di carica è, per definizione, $Q = \lambda L$, mentre la sua 
	posizione è pari a
	$$
	\frac{1}{\lambda L} \int_{0}^{L} \vec{r}\lambda(\vec{r})dl = \frac{L}{2}.
	$$
	Per quanto detto prima, il lavoro necessario a spostare il centro di carica
	è pari al lavoro necessario per spostarlo di un tratto verticale di lunghezza
	$L/2$, cioè:
	$$
	W_R = -QE\frac{L}{2} = - \lambda L E \frac{L}{2} = -\lambda E \frac{L^2}{2}
	= \boxed{W_R = -\frac{\lambda \sigma L^2}{4\varepsilon_0}}.
	$$
\end{proof}

\begin{problema}
	In un cilindro infinitamente lungo di raggio $R$, uniformemente carico
	con densità di carica $\rho$, è praticato, fuori asse, un foro di raggio $a$
	il cui centro è distante $d$ dal centro del cilindro.
	
	Calcolare il campo elettrico all'interno del foro.
	\begin{figure}
		\centering
		\begin{asy}
			import graph;
			import markers;
			texpreamble("\let\oldhat\hat
			\renewcommand{\vec}[1]{\mathbf{#1}}
			\renewcommand{\hat}[1]{\oldhat{\mathbf{#1}}}");
			size(8cm);
			real xmin=-8,xmax=8;
			real ymin=-8,ymax=8;
			xaxis(Label("\small $x$",position=EndPoint,align=SE),
			xmin, xmax, arrow=EndArrow, NoTicks);
			yaxis(Label("\small $y$",position=EndPoint,align=NW),
			ymin, ymax, arrow=EndArrow, NoTicks);
			path cyl_proj = circle((0,0),6);
			path cyl_small_proj = circle((4,0),1);
			filldraw(cyl_proj^^cyl_small_proj,evenodd+grey+opacity(0.65));
			label("$\rho$",(3*sqrt(2),3*sqrt(2)),N);
			label("$-\rho$",(4+sqrt(2)/2,sqrt(2)/2),N);
			pair P = (3.5,-0.5);
			dot(P);
			draw(Label("\small $\vec{r^+}$",position=MidPoint),
			(0,0)--P,linewidth(1bp),Arrow(10bp,12));
			draw(Label("\small $\vec{r^-}$",position=MidPoint,align=E),
			(4,0)--P,linewidth(1bp),Arrow(10bp,12));
			draw(Label("\small $\vec{r^+-r^-}$",position=MidPoint,align=N),
			(0,0)--(4,0),linewidth(1bp),Arrow(10bp,12));	
			draw((4,0)--(4,-4),dashed);
			draw(Label("\small $d$"),(0,-4)--(4,-4),Arrows);
		\end{asy}
		\caption{Cilindro forato}
		\label{fig:cilindro_forato}
	\end{figure}
\end{problema}
\begin{proof}
	Considerata la geometria del sistema, possiamo:
	\begin{enumerate}
		\item ridurre il problema a due dimensioni, e rappresentare il cilindro
		-- visto dall'alto -- come un cerchio al cui interno è presente un altro
		cerchio più piccolo, fuori asse;
		\item supporre che $\rho > 0$;
		\item supporre che i centri dei due cerchi siano entrambi sull'asse $x$,
	\end{enumerate}
	senza perdere di generalità. La situazione è schematizzata in Figura
	\ref{fig:cilindro_forato}. Il nostro sistema può essere ora immaginato come
	la sovrapposizione di due cilindri: uno di raggio $R$ e densità di carica $\rho$,
	e uno di raggio $a$ e densità di carica $-\rho$. Sappiamo che il campo all'interno
	di un cilindro carico uniformemente è
	\begin{equation}
	\vec{E} = \frac{\rho}{2\varepsilon_0}\vec{r},
	\end{equation}
	dove $\vec{r}$ è il vettore posizione di un generico punto all'interno del cilindro
	rispetto all'asse, ed è perpendicolare a quest'ultimo. Indichiamo poi con 
	$$
	\vec{E^+} = \frac{\rho}{2\varepsilon_0}\vec{r^+}
	$$
	il campo generato dal cilindro grande, e con
	$$
	\vec{E^-} = -\frac{\rho}{2\varepsilon_0}\vec{r^-}
	$$
	il campo generato da quello piccolo; dal principio di sovrapposizione il campo totale 
	sarà quindi
	$$
	\vec{E} = \vec{E^+} + \vec{E^-} = 
	\frac{\rho}{2\varepsilon_0}(\vec{r^+}-\vec{r^-}) =
	\boxed{\frac{\rho}{2\varepsilon_0}d\hat{x}}
	$$
\end{proof}

\begin{problema}
	Date tre cariche $q_1,\,q_2,\,q_3$ disposte casualmente nello spazio, trovare
	in un dato punto $P$ i potenziali $V_1,\,V_2,\,V_3$ che generano.
	
	Si sostituiscano ora $q_1,\,q_2,\,q_3$ con altre tre cariche
	$q_1',\,q_2',\,q_3'$ e si ricalcolino i nuovi potenziali $V_1',\,V_2',\,V_3'$.
	\begin{enumerate}
		\item Si dimostri che
		$$
		q_1V_1' + q_2V_2' + q_3V_3' =
		q_1'V_1 + q_2'V_2 + q_3'V_3.
		$$
		In generale, per $N$ cariche, vale che
		$$
		\sum_{i=1}^{N} (q_iV_i' - q_i'V_i) = 0
		$$
		(Teorema della reciprocità di Green).
		\item Estendere il teorema al caso di un conduttore perfetto su cui è
		depositata una carica $Q$ e al potenziale del conduttore.
	\end{enumerate}
\end{problema}
\begin{proof}
	\begin{enumerate}
		\item Fissiamo un sistema di riferimento e indichiamo con
		$\vec{r_1},\,\vec{r_2},\,\vec{r_3}$ i vettori posizione -- rispettivamente --
		di $q_1,\,q_2,\,q_3$ (che coincidono anche con quelli di $q_1',\,q_2',\,q_3'$).
		Detto $\vec{r_P}$ il vettore posizione del generico punto $P$, valgono:
		\begin{eqnarray}
		\displaystyle
		V_1 = \frac{1}{4\pi\varepsilon_0} \frac{q_1}{||\vec{r_P}-\vec{r_1}||},
		& \displaystyle
		V_2 = \frac{1}{4\pi\varepsilon_0} \frac{q_2}{||\vec{r_P}-\vec{r_2}||},
		& \displaystyle
		V_3 = \frac{1}{4\pi\varepsilon_0} \frac{q_3}{||\vec{r_P}-\vec{r_3}||},
		\\ \displaystyle
		V_1' = \frac{1}{4\pi\varepsilon_0} \frac{q_1'}{||\vec{r_P}-\vec{r_1}||},
		& \displaystyle
		V_2' = \frac{1}{4\pi\varepsilon_0} \frac{q_2'}{||\vec{r_P}-\vec{r_2}||},
		& \displaystyle
		V_3' = \frac{1}{4\pi\varepsilon_0} \frac{q_3'}{||\vec{r_P}-\vec{r_3}||},
		\end{eqnarray}
		
		Dalle equazioni scritte sopra, notiamo
		\begin{equation}
		q_iV_i' = q_i'V_i
		\end{equation}
		$\forall \, i = 1,\,2,\,3$, ossia
		\begin{equation}
		\sum_{i=1}^{3} (q_iV_i' - q_i'V_i) = 0
		\label{eq:green_discrete}
		\end{equation}
		che prova la tesi. L'estensione al caso di $N$ cariche è immediata.
		\item Basta generalizzare l'equazione (\ref{eq:green_discrete}) (per $N$ cariche
		puntiformi) a due configurazioni con distribuzione superficiale di carica,
		\footnote{In questo caso, si tratta dello \emph{stesso} conduttore caricato in due
		modi diversi. Per la generalizzazione a due corpi \emph{diversi} con distribuzioni
		di carica generiche, si veda più avanti.}
		chiamiamole $\sigma_1$ e $\sigma_2$, sostituendo $\sum$ con $\int$ e $q_i$ con 
		$dq$.
		
		Abbiamo dunque:
		$$
		\int_{S_1} \sigma_1 V_2 dS_1 = \int_{S_2} \sigma_2 V_1 dS_2
		$$
		che, poiché i potenziali sono costanti e 
		$\displaystyle_S \int \sigma dS = Q$, si riduce a:
		$$
		Q_1V_2 = Q_2V_1.
		$$
		Definendo
		$$
		C \doteqdot \frac{Q}{V},
		$$
		possiamo riscrivere il risultato come:
		$$
		\boxed{C_1 = C_2.}
		$$
		Il risultato ci dice che la quantità $C$ non dipende da \emph{come} carichiamo
		un dato conduttore, ma è una \emph{proprietà specifica} del conduttore stesso
		che rimane costante. Tale proprietà si chiama \emph{capacità elettrica}.
	\end{enumerate}
\end{proof}

\begin{problema}
	Generalizzare l'equazione (\ref{eq:green_discrete})
	a due distribuzioni di carica qualsiasi
	\footnote{
	Il problema proposto in precedenza fa riferimento a due distribuzioni di carica con la
	\emph{stessa} geometria, ed in particolare due distribuzioni superficiali. Qui si
	stanno invece considerando due distribuzioni volumetriche di forma
	qualsiasi, anche molto diversa fra loro. Lo scopo è di dimostrare l'uguaglianza 
	nel caso più generale di due corpi generici, non necessariamente per uno 
	stesso conduttore caricato in due modi differenti.
	}
	(si veda anche Problema 3.50 \cite{griffiths}).
\end{problema}
\begin{proof}
	Dobbiamo provare che, data una distribuzione di carica $\rho_1(\vec{r})$
	che genera un potenziale $V_1$, e una distribuzione di carica $\rho_2(\vec{r})$
	che genera un potenziale $V_2$, vale la relazione:\footnotemark
	\begin{equation}
	\int\limits_{\text{spazio}} \rho_1V_2d\tau =
	\int\limits_{\text{spazio}} \rho_2V_1d\tau
	\end{equation}
	\footnotetext{
	$\displaystyle \int\limits_{\text{spazio}}$ indica l'integrale su tutto 
	lo spazio.
	}
	
	Partiamo dall'identità vettoriale
	\begin{equation}
	\nabla \cdot (f\vec{A}) = f(\nabla \cdot \vec{A}) + \vec{A} \cdot (\nabla f)
	\label{eq:grad_scalar_prod}
	\end{equation}
	e applichiamola con $f\vec{A} = V_1(\nabla V_2)$, ottenendo:
	$$
	\nabla \cdot (V_1\nabla V_2) =
	V_1\nabla \cdot \nabla V_2 + \nabla V_1 \cdot \nabla V_2
	$$
	Ora, usando il fatto che $\vec{E_1} = -\nabla V_1$, $\vec{E_2} = -\nabla V_2$
	e $\nabla \cdot \vec{E_2} = \rho_2/\varepsilon_0$, otteniamo:
	\begin{equation}
	\vec{E_1} \cdot \vec{E_2} =
	\nabla \cdot (V_1\nabla V_2) + V_1\frac{\rho_2}{\varepsilon_0}
	\end{equation}
	Integrando su tutto lo spazio e usando il teorema della divergenza:
	\begin{align*}
	\int\limits_{\text{spazio}} \vec{E_1} \cdot \vec{E_2} d\tau &=
	\int\limits_{\text{spazio}} \nabla \cdot (V_1\nabla V_2) d\tau +
	\frac{1}{\varepsilon_0} \int\limits_{\text{spazio}} \rho_2 V_1 d\tau \\
	&= \cancelto{0}{\int\limits_{\text{spazio}} V_1\nabla V_2 \cdot d\vec{S}} +
	\frac{1}{\varepsilon_0} \int\limits_{\text{spazio}} \rho_2 V_1 d\tau
	\end{align*}
	dove il primo integrale è nullo perché è un integrale di superficie
	su tutto lo spazio. Rimane quindi:
	\begin{equation}
	\int\limits_{\text{spazio}} \vec{E_1} \cdot \vec{E_2} d\tau =
	\frac{1}{\varepsilon_0} \int\limits_{\text{spazio}} \rho_2 V_1 d\tau
	\label{eq:green_part1}
	\end{equation}
	Analogamente, applicando (\ref{eq:grad_scalar_prod}) con
	$f\vec{A} = V_2(\nabla V_1)$, otteniamo:
	\begin{equation}
	\int\limits_{\text{spazio}} \vec{E_1} \cdot \vec{E_2} d\tau =
	\frac{1}{\varepsilon_0} \int\limits_{\text{spazio}} \rho_1 V_2 d\tau
	\label{eq:green_part2}
	\end{equation}
	Confrontando (\ref{eq:green_part1}) e (\ref{eq:green_part2}), otteniamo
	infine
	$$
	\int\limits_{\text{spazio}} \rho_1V_2d\tau =
	\int\limits_{\text{spazio}} \rho_2V_1d\tau,
	$$
	che prova la tesi.
\end{proof}

\begin{problema}
	Una carica $q > 0 $ è posta in un punto $P$ di fronte ad un conduttore collegato a terra.
	\begin{enumerate}
		\item Calcolare la carica $q_{\mathrm{in}}$ indotta sul conduttore dalla carica $q$
		applicando il Teorema di reciprocità, e supponendo che lo stesso conduttore, carico 
		ad un potenziale $V_C$, generi nel punto $P$ il potenziale $V_P$ in assenza di carica.
		\item Se il generico conduttore è sostituito da una sfera conduttrice di raggio $R$ il
		cui centro è distante $D$ dal punto $P$, si determini $q_{\mathrm{in}}$ in funzione di
		$q$, $R$ e $D$ alle stesse condizioni del punto precedente.
	\end{enumerate}
\end{problema}
\begin{proof}
	\begin{enumerate}
		\item Applicando il Teorema di reciprocità, abbiamo:
		$$
		|V_Cq_{\mathrm{in}}| = |V_Pq|
		$$
		da cui:
		$$
		\boxed{q_{\mathrm{in}} = - \left\lvert \frac{V_P}{V_C} \right\rvert q.}
		$$
		\item La sfera conduttrice è carica ad un potenziale
		$$
		V_C = k_e \frac{q_{\mathrm{in}}}{R},
		$$
		mentre il potenziale nel punto $P$ è pari a
		$$
		V_C = k_e \frac{q_{\mathrm{in}}}{D}
		$$
		(immaginando $q_{\mathrm{in}}$ come puntiforme, al centro della sfera). Applicando la
		relazione trovata al punto precedente, abbiamo:
		$$
		q_{\mathrm{in}} = 
		- \left\lvert \frac{V_P}{V_C} \right\rvert q =
		\boxed{- \frac{R}{D}q.}
		$$
	\end{enumerate}
\end{proof}

\begin{problema}
	Sia
	$$
	\vec{A} = -y \hat{i} + x \hat{j}
	$$
	un campo vettoriale.
	\begin{enumerate}
		\item Calcolare l'equazione delle linee di campo di $\vec{A}$ sul piano $xy$.
		\item Calcolare il rotore di $\vec{A}$.
		\item Calcolare la circuitazione di $\vec{A}$ lungo la linea chiusa di equazione
		$x^2 + y^2 = 1$.
		\item Dimostrare la validità del Teorema di Stokes applicata alla superficie piana
		delimitata dalla curva $x^2 + y^2 = 1$.
	\end{enumerate}
\end{problema}
\begin{proof}
	\begin{enumerate}
		\item Partiamo dalla relazione
		$$
		\vec{A} = -y \hat{i} + x \hat{j}.
		$$
		Vale che:
		$$
		||\vec{A}||^2 = (-y)^2 + (x)^2 = x^2 + y^2.
		$$
		Definendo $||\vec{A}||^2 \doteqdot R$, l'equazione delle linee di campo è chiaramente
		l'equazione di una circonferenza di raggio $R$:
		$$
		\boxed{x^2 + y^2 = R,}
		$$
		con $R \in [0,\,+\infty)$.
		\item Dalla definizione,
		$$
		\nabla \times \vec{A} =
		\det\left(
		\begin{matrix}
		\hat{i} & \hat{j} & \hat{k}\\
		\frac{\partial}{\partial x} & \frac{\partial}{\partial y} & \frac{\partial}{\partial z}\\
		-y & x & 0
		\end{matrix}
		\right) =
		-\cancelto{0}{\frac{\partial x}{\partial z}}\hat{i} - 
		\cancelto{0}{\frac{\partial y}{\partial z}}\hat{j} +
		2\hat{k} = \boxed{2\hat{k}.}
		$$
		\item Notiamo che $\vec{A}$ è tangente in ogni punto alla linea lungo la quale
		stiamo calcolando la circuitazione. Inoltre, possiamo pensare di percorrere
		la curva $x^2+y^2=1$ in senso antiorario (come da convenzione), mentre $\vec{A}$
		percorre la stessa curva ma in senso orario. Quindi:
		\begin{align*}
		\int \vec{A} \cdot d\vec{l} 
		&= \int_l (-ydx + xdy)\\
		&= \int_{0}^{2\pi}
		[-sin\vartheta (-\sin\vartheta d\vartheta) +\cos\vartheta (cos\vartheta d\vartheta)]\\
		&= \int_{0}^{2\pi} d\vartheta = \boxed{2\pi.}
		\end{align*}
		\item Vogliamo verificare la validità del Teorema di Stokes, ossia:
		$$
		\int_{S} (\nabla \times \vec{A}) \cdot d\vec{S} 
		= \int_{l} \vec{A} \cdot d\vec{S}.
		$$
		Calcoliamo quindi il primo termine dell'uguaglianza:
		$$
		\int_{S} (\nabla \times \vec{A}) \cdot d\vec{S} 
		= (\nabla \times \vec{A}) \cdot \int_{S} d\vec{S} 
		= (\nabla \times \vec{A}) \cdot \vec{S}
		= \boxed{2\pi.}
		$$
		Il Teorema di Stokes è dunque verificato in questo caso.
	\end{enumerate}
\end{proof}

\begin{problema}
	Dati i campi:
	\begin{align*}
	\vec{A} &= (x+y)\hat{i} + (-x+y)\hat{j} - 2z\hat{k}\\
	\vec{B} &= 2y\hat{i} + (2x+3z)\hat{j} + 3y\hat{k}\\
	\vec{C} &= (x^2-z^2)\hat{i} + 2\hat{j} + 2xz\hat{k}
	\end{align*}
	calcolare per ciascuno di essi
	\begin{enumerate}
		\item la divergenza;
		\item il rotore.
	\end{enumerate}
	Uno solo di questi campi è un campo elettrostatico. Si determini la sua funzione
	potenziale $V = V(x,\,y,\,z)$ e si verifichi che il suo gradiente cambiato di segno
	coincide con il suddetto campo.
\end{problema}
\begin{proof}
	\begin{enumerate}
		\item
		\begin{align*}
		\nabla \cdot \vec{A}
		&= \frac{\partial (x+y)}{\partial x} +
		\frac{\partial (-x+y)}{\partial y} +
		\frac{\partial (-2z)}{\partial z}\\
		&= 1 + 1 - 2 = \boxed{0}.
		\end{align*}
		\begin{align*}
		\nabla \cdot \vec{B}
		&= \frac{\partial 2y}{\partial x} +
		\frac{\partial (2x+3z)}{\partial y} +
		\frac{\partial 3y}{\partial z}\\
		&= 0 + 0 + 0 = \boxed{0}.
		\end{align*}
		\begin{align*}
		\nabla \cdot \vec{C}
		&= \frac{\partial (x^2-z^2)}{\partial x} +
		\frac{\partial 2}{\partial y} +
		\frac{\partial 2xz}{\partial z}\\
		&= 2x + 0 + 2x = \boxed{4x}.
		\end{align*}
		\item
		\begin{align*}
		\nabla \times \vec{A}
		&= \det\left(
		\begin{matrix}
		\hat{i} & \hat{j} & \hat{k}\\
		\frac{\partial}{\partial x} & \frac{\partial}{\partial y} & \frac{\partial}{\partial z}\\
		(x+y) & (-x+y) & -2z
		\end{matrix}
		\right)\\
		&= \cancelto{0}{\hat{i}(0-0)} - \cancelto{0}{\hat{j}(0-0)} + \hat{k}(-1-1)
		= \boxed{-2\hat{k}}.
		\end{align*}
		\begin{align*}
		\nabla \times \vec{B}
		&= \det\left(
		\begin{matrix}
		\hat{i} & \hat{j} & \hat{k}\\
		\frac{\partial}{\partial x} & \frac{\partial}{\partial y} & \frac{\partial}{\partial z}\\
		2y & (2x+3z) & 3y
		\end{matrix}
		\right)\\
		&= \cancelto{0}{\hat{i}(3-3)} - \cancelto{0}{\hat{j}(0-0)} + \cancelto{0}{\hat{k}(2-2)}
		= \boxed{\vec{0}}.
		\end{align*}
		\begin{align*}
		\nabla \times \vec{C}
		&= \det\left(
		\begin{matrix}
		\hat{i} & \hat{j} & \hat{k}\\
		\frac{\partial}{\partial x} & \frac{\partial}{\partial y} & \frac{\partial}{\partial z}\\
		(x^2-z^2) & 2 & 2xz 
		\end{matrix}
		\right)\\
		&= \cancelto{0}{\hat{i}(0-0)} - \hat{j}(2z+2z) + \cancelto{0}{\hat{k}(0-0)}
		= \boxed{-4z\hat{j}}.
		\end{align*}
	\end{enumerate}
	Il campo elettrostatico è chiaramente $\vec{B}$, perché è l'unico ad avere rotore nullo.
	La relazione che intercorre fra il campo elettrico $\vec{B}$ e il suo potenziale $V$ è
	$$
	\vec{B} = -\nabla V,
	$$
	quindi per ottenere $V$ bisogna risolvere il sistema:
	$$
	\begin{cases}\medskip
	\dfrac{\partial V}{\partial x} = -2y\\\medskip
	\dfrac{\partial V}{\partial y} = -2x - 3z\\
	\dfrac{\partial V}{\partial z} = -3y
	\end{cases}
	$$
	Integrando,
	$$
	\begin{cases}
	V = -2xy + c_1(y,\,z)\\
	V = -2xy - 3yz + c_2(x,\,z)\\
	V = -3yz + c_3(x,\,y)
	\end{cases}
	$$
	Usando -- ad esempio -- le prime due equazioni, otteniamo la condizione
	$$
	c_1(y,\,z) = -3yz + c_2(x,\,z).
	$$
	Ora, poiché $c_1$ è funzione di $y$ e $z$ e $c_2$ è funzione di $x$ e $z$, l'unico modo
	di soddisfare la condizione è richiedere che $c_2$ sia nulla rispetto a $x$, o in altri
	termini che $c_2$ sia funzione solo di $z$ e non di $x$. Usando questo fatto e la terza
	equazione del sistema, otteniamo poi
	$$
	c_3(x,\,y) = -2xy + c_2(z).
	$$
	Di nuovo, per soddisfare la condizione, bisogna richiedere che $c_2$ sia nulla rispetto
	a $z$, ossia $c_2 = 0$. Riassumendo,
	$$
	\begin{cases}
	c_1(y,\,z) = -3yz\\
	c_2(x,\,z) = 0\\
	c_3(x,\,y) = -2xy
	\end{cases}
	\qquad \Longrightarrow \qquad
	\boxed{V = - 2xy - 3yz}.
	$$
	La relazione $\vec{B} = -\nabla V$ è facile da verificare. Vale:
	$$
	-\nabla V = \nabla (-V) = \nabla(2xy + 3yz)
	= (2y,\,2x+3z,\,3y) \equiv \vec{B}.
	$$
	
	\underline{Soluzione alternativa.} \`E possibile ottenere $V$ usando la relazione
	$$
	V(P) = {\int\limits_l}^P_{V=0} \vec{B} \cdot d\vec{l},
	$$
	dove $l$ è un cammino arbitrario dal punto in cui poniamo $V=0$ fino al punto 
	$P = (x_0,\,y_0,\,z_0)$ in cui vogliamo calcolare il potenziale. 
	Esprimendo $dl$ come $(dx,\,dy,\,dz)$ e
	scegliendo come punto iniziale $(0,\,0,\,0)$ abbiamo:
	$$
	V(P) = {\int\limits_l}_O^P [2ydx + (2x+3z)dy + 3ydz].
	$$
	Scegliamo -- ad esempio -- il cammino riportato in Figura \ref{fig:cammino_potenziale}.
		\begin{figure}[H]
			\centering
			\begin{asy}[height=6cm,inline=true,attach=false,viewportwidth=\linewidth]
				import three;
				import graph3;
				import solids;
				currentprojection=obliqueX(-45);
				texpreamble("\let\oldhat\hat
				\renewcommand{\vec}[1]{\mathbf{#1}}
				\renewcommand{\hat}[1]{\oldhat{\mathbf{#1}}}");
				size3(6cm);
				real xmin=0,xmax=6;
				real ymin=0,ymax=6;
				real zmin=0,zmax=6;
				xaxis3(Label("\small $x$",position=EndPoint,align=SE),
				xmin, xmax, arrow=EndArrow3, NoTicks3);
				yaxis3(Label("\small $y$",position=EndPoint,align=NW),
				ymin, ymax, arrow=EndArrow3, NoTicks3);
				zaxis3(Label("\small $z$",position=EndPoint,align=NW),
				zmin, zmax, arrow=EndArrow3, NoTicks3);
				triple O=(0,0,0);
				triple P1=(4,0,0);
				triple P2=(4,4,0);
				triple P3=(4,4,4);
				draw(Label("\small I"),O--P1,linewidth(1.5bp));
				draw(Label("\small II"),P1--P2,linewidth(1.5bp));
				draw(Label("\small III"),P2--P3,linewidth(1.5bp));
				dot(Label("\small $P=(x_0,\,y_0,\,z_0)$"),P3);
			\end{asy}
			\caption{Cammino arbitrario tra $0$ e $P$}
			\label{fig:cammino_potenziale}
		\end{figure}
		Mettendo un segno ``$-$'' davanti al $dy$ nel tratto II,
		e davanti al $dz$ nel tratto III, perché gli spostamenti sono in verso opposto
		agli assi, si ha:
		\begin{align*}
		V(P) &= {\int\limits_l}^P_{V=0} \vec{B} \cdot d\vec{l}
		= \int_{I} \vec{B} \cdot d\vec{l} 
		+ \int_{II} \vec{B} \cdot d\vec{l} 
		+ \int_{III} \vec{B} \cdot d\vec{l}\\
		&= \int\limits_{(0,0,0)}^{(x_0,0,0)} 
		[\cancel{2 \cdot 0 \cdot dx} 
		+ \cancel{(2x+3 \cdot 0) \cdot 0} 
		+ \cancel{3 \cdot 0 \cdot 0}] +\\
		&+ \int\limits_{(x_0,0,0)}^{(x_0,y_0,0)} 
		[\cancel{2 y \cdot 0} 
		- (2x+3 \cdot 0) \cdot dy 
		+ \cancel{3 y \cdot 0}] +\\
		&+ \int\limits_{(x_0,y_0,0)}^{(x_0,y_0,z_0)} 
		[\cancel{2 y \cdot 0} 
		+ \cancel{(2x+3 y) \cdot 0}
		- 3 y dz]\\
		&= \left.-2xy\right\rvert_{(x_0,0,0)}^{(x_0,y_0,0)}
		+ \left.-3yz\right\rvert_{(x_0,y_0,0)}^{(x_0,y_0,z_0)}
		= - 2x_0y_0 - 3y_0z_0,
		\end{align*}
		ossia
		$$
		\boxed{V(x,\,y,\,z) = -2xy - 3yz}.
		$$
\end{proof}

\begin{problema}
Si calcolino:
	\begin{enumerate}
		\item il campo elettrico fra due cilindri coassiali infinitamente lunghi aventi 
		raggi $a$ e $b$ (Figura \ref{fig:condensatore_cilindrico_infinito}), sapendo che la 
		carica per unità di lunghezza sul cilindro interno è $+\lambda$ mentre quella sul
		cilindro esterno è $-\lambda$;
		\item la differenza di potenziale $V_A - V_B$ fra i due cilindri.
	\end{enumerate}
	\begin{figure}[H]
			\centering
			\begin{asy}
				texpreamble("\let\oldhat\hat
				\renewcommand{\vec}[1]{\mathbf{#1}}
				\renewcommand{\hat}[1]{\oldhat{\mathbf{#1}}}");
				size(4cm);
				draw(Label("\small $+\lambda$",p=black),circle((0,0),1),invisible);
				label("\small $A$",rotate(45)*(1,0),E);
				filldraw(circle((0,0),1),grey+opacity(0.3));
				draw(Label("\small $-\lambda$",p=black),circle((0,0),3.3),invisible);
				label("\small $B$",rotate(45)*(3.3,0),E);
				filldraw(circle((0,0),3.3)^^circle((0,0),3),evenodd+grey+opacity(0.3));
				draw(Label("\small $a$",position=MidPoint,align=NNW),(0,0)--rotate(45)*(-1,0));
				draw(Label("\small $b$",position=MidPoint,align=2E),(0,0)--rotate(-45)*(3,0));
			\end{asy}
			\caption{Cilindri coassiali infiniti}
			\label{fig:condensatore_cilindrico_infinito}
		\end{figure}
\end{problema}
\begin{proof}
	\begin{enumerate}
		\item Per ragioni di simmetria, sappiamo che fra i due cilindri il campo elettrico
		 $\vec{E}$ è radiale, ed è diretto dal cilindro interno verso quello esterno.
		 Possiamo quindi calcolarne facilmente il modulo usando il Teorema di Gauss.
		 A tale scopo, consideriamo una superficie gaussiana di forma cilindrica, con
		 raggio $a < r < b$ e altezza $h$, da cui:
		 $$
		 \begin{cases}\medskip
		 \displaystyle
		 \oint\limits_{S} \vec{E}(r) \cdot d\vec{S} = E(r)S = E(r)2\pi rh\\
		 \dfrac{q_{\mathrm{in}}}{\varepsilon_0} = \dfrac{\lambda h}{\varepsilon_0}
		 \end{cases}
		 \qquad \Longrightarrow \qquad
		 \boxed{
		 \vec{E}(\vec{r}) = \frac{\lambda}{2\pi\varepsilon_0}\frac{1}{r}\hat{r}
		 }
		 $$
		 \item La relazione generale che lega $\vec{E}$ e $V$ è $\vec{E} = -\nabla V$. In
		 questo caso, però, $\vec{E}$ è solo funzione della distanza dall'asse, quindi
		 possiamo scrivere che
		 $$
		 E(r) = -\frac{dV}{dr}.
		 $$
		 Integrando,
		 $$
		 V(b) - V(a) = -\int_{a}^{b} E(r)dr 
		 = -\frac{\lambda}{2\pi\varepsilon_0}\left.\ln(r)\right\rvert_{a}^{b}
		 =-\frac{\lambda}{2\pi\varepsilon_0}\ln\left(\frac{b}{a}\right).
		 $$
		 Quindi
		 $$
		 \boxed{
		 V_A - V_B = \frac{\lambda}{2\pi\varepsilon_0}\ln\left(\frac{b}{a}\right)
		 }
		 $$
	\end{enumerate}
\end{proof}

\begin{problema}
	Tre lastre conduttrici piane sono disposte parallelamente tra loro come in Figura
	\ref{fig:tre_lastre}. Le piastre hanno area $A$ e le distanze $x_1$ e $x_2$ sono
	infinitesime rispetto a $\sqrt{A}$. La piastra superiore è connessa a quella
	inferiore tramite un filo conduttore. Sulla piastra centrale è depositata uniformemente
	una carica $Q$.
	
	Determinare le cariche indotte $Q_1$ e $Q_2$ rispettivamente sulla piastra superiore e
	su quella inferiore, note le distanze $x_1$ e $x_2$ e trascurando gli effetti del campo
	ai bordi delle piastre.
	\begin{figure}[H]
			\centering
			\begin{asy}
				texpreamble("\let\oldhat\hat
				\renewcommand{\vec}[1]{\mathbf{#1}}
				\renewcommand{\hat}[1]{\oldhat{\mathbf{#1}}}");
				size(6cm);
				draw(Label("\small $A$",2N),(14,6)--(14,5)--(0,5)--(0,6)--cycle);
				draw(Label("\small $B$",2N),(14,1)--(14,0)--(0,0)--(0,1)--cycle);
				draw(Label("\small $C$",2N),(14,-7)--(14,-8)--(0,-8)--(0,-7)--cycle);
				draw((14,5.5)--(16,5.5)--(16,-7.5)--(14,-7.5));
				label("\small $Q_1$",(7,5.5),NoAlign);
				label("\small $Q$",(7,0.5),NoAlign);
				label("\small $Q_2$",(7,-7.5),NoAlign);
				draw(Label("\small $x_1$",align=Center,filltype=UnFill),
				(-1,1)--(-1,5),arrow=Arrows,bar=Bars);
				draw(Label("\small $x_2$",align=Center,filltype=UnFill),
				(-1,0)--(-1,-7),arrow=Arrows,bar=Bars);
			\end{asy}
			\caption{Tre lastre conduttrici parallele}
			\label{fig:tre_lastre}
		\end{figure}
\end{problema}
\begin{proof}
	Il calcolo delle cariche indotte $Q_1$ e $Q_2$ equivale al calcolo delle distribuzioni
	superficiali di carica $\sigma_1$ e $\sigma_2$, che d'ora in poi saranno le nostre
	incognite.
	
	Usando il Principio di sovrapposizione, il Teorema degli elementi corrispondenti 
	e il fatto che il potenziale su $A$ e
	$C$ deve essere lo stesso perché sono collegate da un filo (il potenziale sulla
	superficie di un conduttore è costante), possiamo impostare il sistema
	$$
	\begin{cases}
	\medskip
	E_1 = \dfrac{\sigma}{2\varepsilon_0} - \dfrac{\sigma_1}{2\varepsilon_0}\\\medskip
	E_2 = \dfrac{\sigma}{2\varepsilon_0} - \dfrac{\sigma_2}{2\varepsilon_0}\\
	E_1x_1 = E_2x_2\\
	Q_1 + Q_2 = -Q
	\end{cases}
	$$
	\`E un sistema di 4 equazioni in 4 incognite ($E_1,\,E_2,\,\sigma_1,\,\sigma_2$), perché
	le $\sigma_i$ sono legate alle $Q_i$ dalla relazione
	$$
	\sigma_i = \frac{Q_i}{A}.
	$$
	Risolvendo, abbiamo:
	$$
	\begin{cases}
	\medskip
	E_1 = \dfrac{3\sigma}{2\varepsilon_0}\dfrac{x_2}{x_1+x_2}\\\medskip
	E_2 = \dfrac{3\sigma}{2\varepsilon_0}\dfrac{x_1}{x_1+x_2}\\\medskip
	\boxed{\sigma_1 = \dfrac{x_1-2x_2}{x_1+x_2}\sigma}\\\medskip
	\boxed{\sigma_2 = \dfrac{x_2-2x_1}{x_1+x_2}\sigma}\\
	\end{cases}
	$$
\end{proof}

\begin{problema}
	Lo strato sottile semisferico illustrato in Figura \ref{fig:emisfero_carico} ha raggio
	$R$ ed è uniformemente carico. Nota la carica $Q$ distribuita su di esso, si calcolino:
	\begin{enumerate}
		\item il potenziale lungo tutto l'asse $z$ ($z>R$ e $z<R$);
		\item il campo elettrico lungo tutto l'asse $z$.
	\end{enumerate}
	\begin{figure}%[H]
		\centering
		\begin{asy}[height=6cm,inline=true,attach=false,viewportwidth=\linewidth]
			import three;
			import graph3;
			import solids;
			currentprojection=orthographic(2,2,1);
			texpreamble("\let\oldhat\hat
			\renewcommand{\vec}[1]{\mathbf{#1}}
			\renewcommand{\hat}[1]{\oldhat{\mathbf{#1}}}");
			size3(8cm);
			real xmin=-8,xmax=8;
			real ymin=-8,ymax=8;
			real zmin=-4,zmax=10;
			zaxis3(Label("\small $z$",position=EndPoint,align=NW),
			zmin, zmax, arrow=EndArrow3, NoTicks3);
			surface plane=surface((-8,-8,0)--(-8,8,0)--(8,8,0)--(8,-8,0)--cycle);
			draw(plane, opacity(0.2)+lightgrey);
			real r=6, h=0; // r=sphere radius; h=altitude section
			real rs=sqrt(r^2-h^2); // section radius
			real ch=180*acos(h/r)/pi;
			path3 arcU=Arc((0,0,0),r,ch,0,0,0,Y,10);
			revolution sphereU=revolution((0,0,0),arcU,Z);
			draw(surface(sphereU), opacity(0.5)+lightgrey);
			dot(Label("\small $P$"),(0,0,8));
			draw(Label("\small $R$",position=MidPoint),(0,0,0)--(0,-6,0));
			draw((0,0,0)--rotate(45,(0,0,1))*(0,-6,0));
			draw((0,0,0)--(0,3*sqrt(2),3*sqrt(2)),linewidth(1.5bp));
			draw(Label("\small $r$"),(0,3*sqrt(2),3*sqrt(2))--(0,0,8),linewidth(1.5bp));
			draw(Label("\small $\vartheta$",align=N),
			(0,0,1)..rotate(-45/2,(1,0,0))*(0,0,1)..rotate(-45,(1,0,0))*(0,0,1));
			draw(Label("\small $\psi$",align=N),
			(0,-1,0)..rotate(45/2,(0,0,1))*(0,-1,0)..rotate(45,(0,0,1))*(0,-1,0));
		\end{asy}
		\caption{Emisfero carico}
		\label{fig:emisfero_carico}
	\end{figure}	
\end{problema}
\begin{proof}
	\begin{enumerate}
		\item Utilizzando gli angoli illustrati in Figura \ref{fig:emisfero_carico}, abbiamo:
		$$
		r = \sqrt{z^2+R^2-2zR\cos\vartheta};
		\qquad
		dS = [(R\sin\vartheta)d\psi](Rd\vartheta) = R^2 \sin\vartheta d\vartheta d\psi.
		$$
		Definita $\sigma = \dfrac{Q}{S} = \dfrac{Q}{2 \pi R^2}$,
		possiamo quindi calcolare il potenziale $V$ direttamente dalla definizione:
		\begin{align*}
		V &= k_e \int_S \frac{1}{r} \sigma dS\\
		&= k_e \int_{\psi=0}^{2\pi} \int_{\vartheta=0}^{\pi/2}
		\frac{\sigma R^2 \sin\vartheta}{\sqrt{z^2+R^2-2zR\cos\vartheta}} d\vartheta d\psi\\
		&= k_e \int_{\psi=0}^{2\pi} d\psi \int_{\vartheta=0}^{\pi/2}
		\frac{\sigma R^2}{2zR}
		\frac{2zR\sin\vartheta}{\sqrt{z^2+R^2-2zR\cos\vartheta}} d\vartheta\\
		&= k_e \int_{\psi=0}^{2\pi} d\psi
		\left[
		\frac{\sigma R^2}{\cancel{2}zR} \cancel{2} \sqrt{z^2+R^2-2zR\cos\vartheta}
		\right]_{0}^{\pi/2}\\
		&= k_e \frac{\sigma R^2}{zR} \left(\sqrt{z^2+R^2}-|z-R|\right)
		\int_{\psi=0}^{2\pi} d\psi\\
		&= -k_e \frac{Q}{\cancel{2\pi R^2}} 
		\frac{\cancel{R^2}}{zR} \left(|z-R|-\sqrt{z^2+R^2}\right) \cancel{2\pi}\\
		&= \boxed{-k_e \frac{Q}{zR} \left(|z-R| - \sqrt{z^2+R^2}\right)}.
		\end{align*}
		Notiamo che
		$$
		V \sim_{\pm\infty} \pm k_e\frac{Q}{z},
		$$
		cioè, a grandi distanze dalla distribuzione, il potenziale va come quello
		di una carica puntiforme $Q$.
		\item
		\begin{align*}
		E &= - |\nabla V| = - \frac{\partial V}{\partial z}\\
		&= \left[
		k_e \frac{Q}{zR} \left(|z-R| - \sqrt{z^2+R^2}\right)
		\right]'\\
		&= k_e Q \left[
		\frac{|z-R|}{zR} - \frac{1}{R}\frac{\sqrt{z^2+R^2}}{z}
		\right]'\\
		&= k_e Q \left[
		\frac{z \cdot \mathrm{sgn}(z-R)-R \cdot \mathrm{sgn}(z-R)}{zR} 
		- \frac{1}{R}\frac{\sqrt{z^2+R^2}}{z}
		\right]'\\
		&= k_e Q \left[
		\frac{\mathrm{sgn}(z-R)}{R} - \frac{\mathrm{sgn}(z-R)}{z} 
		- \frac{1}{R}\frac{\sqrt{z^2+R^2}}{z}
		\right]'\\
		&= k_e Q
		\left(
		0 + \frac{\mathrm{sgn}(z-R)}{z^2}
		- \frac{1}{R}
		\frac{\frac{z}{\sqrt{z^2+R^2}} \cdot z - \sqrt{z^2+R^2} \cdot 1}{z^2} 
		\right)\\
		&= k_e Q
		\left(
		\mathrm{sgn}(z-R) - \frac{1}{R}\frac{\cancel{z^2}-\cancel{z^2}-R^2}{\sqrt{z^2+R^2}}
		\right)\\
		&= \boxed{
		k_e \frac{Q}{z^2} \left(\mathrm{sgn}(z-R) + \frac{R}{\sqrt{z^2+R^2}}\right)
		}.
		\end{align*}
		Notiamo che, come nel caso del potenziale,
		$$
		E \sim_{\pm\infty} \pm k_e\frac{Q}{z^2},
		$$
		cioè, a grandi distanze dalla distribuzione, il campo elettrico va come quello
		di una carica puntiforme $Q$.
	\end{enumerate}
\end{proof}

\begin{problema}
	Un campo elettrico $\vec{E}$ è descritto -- in coordinate cilindriche -- dal potenziale
	$$
	V(r,\,\varphi,\,z) = 80 \varphi \; [V].
	$$
	Calcolare:
	\begin{enumerate}
			\item l'energia immagazzinata nella regione
			$$
			r \in [2,\,4]\, \mathrm{cm}; \qquad
			\varphi \in [0,\,0.2\pi]; \qquad
			z \in [0,\,1] \, \mathrm{m};
			$$
			\item il valore massimo della densità di energia $\mathrm{u}$ nella regione
			specificata;
			\item la differenza di potenziale $V_{AB}$ tra i punti
			$A = (r = 3 \, \mathrm{cm},\,\varphi = 0,\,z = 0)$ e
			$B = (3 \, \mathrm{cm},\,0.2\pi,\,1 \, \mathrm{m})$.
	\end{enumerate}
\end{problema}
\begin{proof}
	\begin{enumerate}
		\item $\vec{E} = - \nabla V = -\dfrac{80}{r} \hat{\varphi} \, \dfrac{C}{m}$.\\
		$\mathrm{u} = \dfrac{\varepsilon_0}{2} E^2 
		= \dfrac{\varepsilon_0}{2}\left(-\dfrac{80}{r}\right)^2 = 
		\dfrac{320}{r^2} \varepsilon_0 \, \dfrac{J}{m^3}$.\\
		$\displaystyle W = \int_{\tau} \mathrm{u} d\tau 
		= 320\varepsilon_0 \int_{\tau} \frac{1}{r^{\cancel{2}^1}}\cancel{r} dr d\varphi dz
		= 320\varepsilon_0 \cdot (2 \times 10^{-1} \pi)
		\cdot 1 \cdot \left.\ln(r)\right\rvert_{2 \times 10^{-2}}^{4 \times 10^{-2}} 
		\, \mathrm{J} = \\
		= 64\pi\ln(2)\varepsilon_0 \, \mathrm{J}
		= \boxed{\dfrac{16\ln(2)}{k_e} \, \mathrm{J}}$.
		\item $\max u \Longleftrightarrow 
		\min r^2 \Longleftrightarrow 
		\min |r| \Longleftrightarrow 
		\min r = 2 \times 10^{-2} \, \mathrm{m}$, da cui\\
		$\max u = 
		\dfrac{320}{\left( 2 \times 10^{-2} \right)^2} \varepsilon_0 \, \dfrac{J}{m^3} =
		\boxed{125 \varepsilon_0 \, \dfrac{J}{m^3}}$
		\item $V_B - V_A = 80(0.2\pi-0) \, \mathrm{V} = \boxed{16\pi \, \mathrm{V}}$.
	\end{enumerate}
\end{proof}

\begin{problema}
	Una carica di densità $\sigma$ è distribuita uniformemente sul triangolo in Figura
	\ref{fig:potenziale_triangolo}; il lato $BC$ ha lunghezza $b$ pari al doppio
	dell'altezza relativa $AH$, di lunghezza $a$. Si calcoli il potenziale sul vertice $A$.
	\begin{figure}[H]
			\centering
			\begin{asy}
				import graph;
				texpreamble("\let\oldhat\hat
				\renewcommand{\vec}[1]{\mathbf{#1}}
				\renewcommand{\hat}[1]{\oldhat{\mathbf{#1}}}");
				size(6cm);
				real xmin=-2,xmax=6;
				real ymin=-4,ymax=4;
				xaxis(Label("\small $x$",position=EndPoint,align=SE),
				xmin, xmax, arrow=EndArrow, NoTicks);
				yaxis(Label("\small $y$",position=EndPoint,align=NW),
				ymin, ymax, arrow=EndArrow, NoTicks);
				pair A=(0,0);
				pair B=(3,3);
				pair C=(3,-3);
				pair H=(3,0);
				path mytriangle = A--B--C--cycle;
				filldraw(mytriangle,lightgrey+opacity(0.7));
				dot("\small $A$",A,W);
				dot("\small $B$",B,NE);
				dot("\small $C$",C,SE);
				dot("\small $H$",H,NE);
				draw(Label("\small $a$",N,position=MidPoint,p=black),A--H,p=invisible);
				draw(Label("\small $a$",E,position=MidPoint,p=black),B--H,p=invisible);
				draw(Label("\small $a$",E,position=MidPoint,p=black),H--C,p=invisible);
			\end{asy}
			\caption{Distribuzione di carica uniforme su un triangolo}
			\label{fig:potenziale_triangolo}
		\end{figure}
\end{problema}
\begin{proof}
	Per calcolare il potenziale in $A$ utilizziamo l'equazione:
	\begin{equation}
	V = \int k_{e}\frac{\sigma}{r}dS,
	\label{eq:potenziale_superficie}
	\end{equation}
	dove $r$ è la distanza del vertice $A$ da un fazzoletto infinitesimo di superficie 
	del triangolo considerato. Osserviamo che $dS = dxdy$ e $r = \sqrt{x^2 + y^2}$. 
	Inoltre $b = 2a $. Facendo riferimento alla Figura \ref{fig:potenziale_triangolo} 
	e sostituendo nell'equazione (\ref{eq:potenziale_superficie}) otteniamo:
	\begin{align*}
	V(A)
	&= k_e \int_0^a dx \int_{-x}^x \frac{\sigma}{\sqrt{x^2 + y^2}} dy \\
	&= k_e\sigma \int_0^a dx \int_{-x}^x \frac{1/x}{\sqrt{1 +( y/x)^2}} dy
	\end{align*}
	dove
	\begin{align*}
	\int_{-x}^x \frac{1/x}{\sqrt{1 +( y/x)^2}} dy
	&= \left.\arsinh\left(\frac{y}{x}\right) \right|_{y=-x}^x \\
	&= 2\arsinh(1) \\
	&= 2\ln\left(1 + \sqrt{2}\right)
	\end{align*}
	dove nell'ultimo passaggio si è utilizzata la relazione:
	$$
	\arsinh(x) = \ln\left(x + \sqrt{x^2+1}\right).
	$$
	Sostituendo ancora:
	\begin{align*}
	V(A)
	&= 2k_e\sigma \int_0^a \ln\left(1 + \sqrt{2}\right) dx \\
	&= 2k_e\sigma\ln\left(1 + \sqrt{2}\right) \int_0^a dx \\
	&= \boxed{2k_e\sigma\ln\left(1 + \sqrt{2}\right)a}
	\end{align*}
	
	\underline{Soluzione alternativa.} Il triangolo può essere visto come una successione
	di fili verticali, ognuno di densità lineare di carica $\lambda = \sigma dx$ e lunghezza
	$2x$, dove $x$ è la posizione del filo che andiamo a considerare. La posizione di un
	elemento infinitesimo $dl$ del filo è $r = \sqrt{x^2+y^2}$, e $dl \equiv dy$. Ogni filo
	dà quindi un contributo al potenziale
	$$
	dV = \int_l \frac{1}{r}\lambda dl = \int_{y=-x}^{x} \frac{1}{r}(\sigma dx) dy.
	$$
	Sommando tutti i contributi al potenziale otteniamo:
	$$
	V(A) = \int dV = 
	\int_{x=0}^{a} \int_{y=-x}^{x} \frac{1}{\sqrt{x^2+y^2}}\sigma dx dy,
	$$
	come già ottenuto in precedenza.
\end{proof}

\begin{problema}
	Si calcoli il momento di dipolo delle seguenti distribuzioni di carica. 
	Sia $\lambda = \mathrm{costante}$ la densità di carica lineare.
	\begin{enumerate}
		\item FIGURA1
		\item FIGURA2
		\item FIGURA3
		\item FIGURA4
	\end{enumerate}
\end{problema}
\begin{proof}
	Il problema può essere risolto in due modi.
	\begin{enumerate}
		\item Si calcolano il centro di carica $C^+$ della distribuzione positiva e
		il centro di carica $C^-$ di quella negativa. La carica totale su ognuna delle due
		distribuzioni è $q > 0$ o $-q$ a seconda della distribuzione, quindi il momento di
		dipolo sarà semplicemente
		$$
		\vec{p} = q\vec{a},
		$$
		dove $\vec{a}$ è il vettore che va da $C^-$ a $C^+$.
		\item Si applica la definizione
		$$
		\vec{p} = \int_l \vec{r}\lambda(\vec{r})dl,
		$$
		dove $\vec{r}$ è il vettore posizione di un elemento infinitesimo di carica, $\lambda$
		la densità lineare di carica in quel punto e $dl$ l'elemento infinitesimo di lunghezza
		della distribuzione.
	\end{enumerate}
	
	Il primo metodo è in questo caso il più veloce, ma se si può applicare è grazie alla
	particolare simmetria delle distribuzioni considerate. Il secondo metodo è più 
	laborioso, ma ha una valenza generale.
	In questa soluzione si è scelto di applicare la definizione.
	
	Passiamo ora ai vari casi. Per comodità in tutta la soluzione, si è posto $\lambda > 0$.
	\begin{enumerate}
		\item 
		\begin{align*}
		\vec{p} &= \int_l \vec{r}\lambda(\vec{r})dl
		= \int_{x=-L}^{L} \vec{x}\lambda(\vec{x})dx \\
		&= \int_{-L}^{0} (x\hat{x})(-\lambda)dx + \int_{0}^{L} (x\hat{x})\lambda dx \\
		&= \lambda \hat{x} \left( -\left.\frac{x^2}{2}\right\rvert_{-L}^{0} 
		+ \left.\frac{x^2}{2}\right\rvert_{0}^{L} \right)\\
		&= \boxed{\lambda L^2 \hat{x}}.
		\end{align*}
		\item 
		\begin{align*}
		\vec{p} &= \int_l \vec{r}\lambda(\vec{r})dl
		= \int_{x=-(L+d)}^{L+d} \vec{x}\lambda(\vec{x})dx \\
		&= \int_{-(L+d)}^{0} (x\hat{x})(-\lambda)dx + \int_{0}^{L+d} (x\hat{x})\lambda dx \\
		&= \lambda \hat{x} \left( -\left.\frac{x^2}{2}\right\rvert_{-(L+d)}^{0} 
		+ \left.\frac{x^2}{2}\right\rvert_{0}^{L+d} \right)\\
		&= \lambda \left[ (L+d)^2 - d^2 \right] \hat{x}
		= \boxed{\lambda L(L+2d)\hat{x}}.
		\end{align*}
		\item 
		\begin{align*}
		\vec{p} &= \int_l \vec{r}\lambda(\vec{r})dl \\
		&= \int_{y=0}^{L} (y\hat{y})\lambda dy + \int_{x=L}^{0} (x\hat{x})(-\lambda)(-dx) \\
		&= \int_{y=0}^{L} y\hat{y}\lambda dy - \int_{x=0}^{L} x\hat{x}\lambda dx \\
		&= \lambda \left[ \frac{L^2}{2}\hat{y} - \frac{L^2}{2}\hat{x} \right]
		= \boxed{\lambda \frac{L^2}{2}(\hat{y}-\hat{x})}.
		\end{align*}
		\item 
		\begin{align*}
		\vec{p} &= \int_l \vec{r}\lambda(\vec{r})dl \\
		&= \int_{\vartheta=0}^{\pi} (R\cos\vartheta\hat{x})\lambda(Rd\vartheta) 
		+ \int_{\vartheta=0}^{\pi} (R\sin\vartheta\hat{y})\lambda(Rd\vartheta) + \\
		&+ \int_{\vartheta=\pi}^{2\pi} (R\cos\vartheta\hat{x})(-\lambda)(Rd\vartheta) 
		+ \int_{\vartheta=\pi}^{2\pi} (R\sin\vartheta\hat{y})(-\lambda)(Rd\vartheta) \\
		&= \left[ \lambda R^2 \sin\vartheta \hat{x} 
		- \lambda R^2 \cos\vartheta \hat{y} \right]_{\vartheta=0}^{\pi}
		- \left[ \lambda R^2 \sin\vartheta \hat{x} 
		- \lambda R^2 \cos\vartheta \hat{y} \right]_{\vartheta=\pi}^{2\pi} \\
		&= \boxed{4 \lambda R^2 \hat{y}}.
		\end{align*}
	\end{enumerate}
\end{proof}

\begin{problema}
	Una linea di trasmissione è costituita da due cavi cilindrici	 di raggio $R$ paralleli
	e infinitamente lunghi i cui centri distano $D$. Su un cavo è distribuita uniformemente
	una densità superficiale di carica $\sigma > 0$, sull'altro una densità superficiale di 
	carica $-\sigma$ (sempre in maniera uniforme). Si calcoli la differenza di potenziale tra
	i due conduttori.
	\begin{figure}[H]
			\centering
			\begin{asy}
				import graph;
				texpreamble("\let\oldhat\hat
				\renewcommand{\vec}[1]{\mathbf{#1}}
				\renewcommand{\hat}[1]{\oldhat{\mathbf{#1}}}");
				size(6cm);
				real xmin=0,xmax=7;
				xaxis(Label("\small $x$",position=EndPoint,align=SE),
				xmin, xmax, arrow=EndArrow, NoTicks);
				path cylinder1 = circle((0,0),1);
				path cylinder2 = circle((5,0),1);
				filldraw(cylinder1,lightgrey+opacity(0.7));
				filldraw(cylinder2,lightgrey+opacity(0.7));
				draw(Label("\small $D$",align=Center,filltype=UnFill),
				(0,-2)--(5,-2),arrow=Arrows,bar=Bars);
				draw(Label("\small $R$",position=MidPoint,align=WNW),
				(0,0)--rotate(-45)*(0,1));
				draw(Label("\small $R$",position=MidPoint,align=WNW),
				(5,0)--rotate(-45,(5,0))*(5,1));
				labelx("\small $0$",0,S);
			\end{asy}
			\caption{Cavi coassiali in sezione}
			\label{fig:cavi_coassiali_sezione}
		\end{figure}
\end{problema}
\begin{proof}
	Siano $\vec{r_1}$ e $\vec{r_2}$ i versori radiali, uscenti dalle due sezioni; $r_{1}$ e
	$r_{2}$ le distanze di un qualsiasi punto dello spazio dai rispettivi centri. 
	Utilizzando il teorema di Gauss si verifica che il campo generato dai due cavi 
	(consideriamo $1$ il cilindro di sinistra) è:
	$$
	\vec{E_{1}} = \frac{\sigma R}{\varepsilon_{0}r_{1}}\vec{r_{1}}, 
	\qquad 
	\vec{E_{2}} = -\frac{\sigma R}{\varepsilon_{0}r_{2}}\vec{r_{2}}.
	$$
	Ora, il campo elettrico è conservativo; quindi possiamo calcolare la differenza di
	potenziale fra le due superfici sul percorso privilegiato che è la congiungente degli
	assi dei cilindri. Orientando tale congiungente con un asse cartesiano $x$ che ha 
	origine nel centro del cilindro $1$, possiamo scrivere il campo \emph{limitatamente}
	alla congiungente come somma dei campi $\vec{E_1}$ ed $\vec{E_2}$:
	$$
	\vec{E_{\mathrm{tot, x}}}(\vec{x})
	= \frac{\sigma R}{\varepsilon_0}\left(\frac{1}{x} + \frac{1}{D-x}\right)\hat{x}.
	$$
	Il vantaggio di limitarci alla congiungente è che qui il campo complessivo è
	parallelo al versore $\hat{x}$, e quindi la relazione generale
	$\displaystyle V(\vec{a}) - V(\vec{b)} = - \int_{b}^{a} \vec{E}(\vec{r}) \cdot d\vec{l} $ 
	si semplifica notevolmente. Otteniamo infatti:
	\begin{align*}
	V_1 - V_2 &= -\int_{x=D-R}^{x=R} E_{\mathrm{tot, x}}(\vec{x}) dx \\
	&= -\int_{x=D-R}^{x=R} 
	\frac{\sigma R}{\varepsilon_0}\left(\frac{1}{x} + \frac{1}{D-x}\right) dx \\
	&= -\frac{\sigma R}{\varepsilon_0}
	\left[ \ln(x) - \ln\left(D-x\right) \right]_{D-R}^{R}\\
	&= 2\frac{\sigma R}{\varepsilon_0}[\ln(D-R)-\ln(R)]
	= \boxed{2\frac{\sigma R}{\varepsilon_0}\ln\left(\frac{D}{R}-1\right)}.
	\end{align*}
\end{proof}


\begin{thebibliography}{9}

\bibitem[Griffiths]{griffiths}
  David J. Griffiths,
  \emph{Introduction to Electrodynamics}.
  Addison-Wesley, 
  4th edition,
  2012.
  \texttt{ISBN-13 978-0-321-85656-2}.

\bibitem[Mencuccini]{mencuccini}
  Corrado Mencuccini, Vittorio Silvestrini,
  \emph{Fisica II: Elettromagnetismo -- Ottica}.
  Liguori Editore, Napoli,
  1988.
  \texttt{ISBN 88-207-1633-X}.

\end{thebibliography}

\end{document}
